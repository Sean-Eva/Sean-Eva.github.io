\documentclass{article}
\usepackage[utf8]{inputenc}
\usepackage[english]{babel}
\usepackage{amsthm}
\usepackage{amssymb}
\usepackage{mathcomp}
\usepackage{amsmath}
\usepackage{natbib}
\usepackage{array}
\usepackage{wrapfig}
\usepackage{multirow}
\usepackage{tabularx}

\newtheorem{ishaan}{Theorem}[section]
\newtheorem{lemma}{Lemma}[section]
\renewcommand\qedsymbol{$\blacksquare$}

\title{Homework 5}
\author{Sean Eva}
\date{March 2021}

\begin{document}

\maketitle

\begin{enumerate}
    \item 
    
    It is trivial to see that there are two eigenvalues for the matrix $A$, $3$ and $2$. In order for this matrix to be diagonalizable, the geometric multiplicity of $\lambda = 2$ would need to be $1$ and the geometric multiplicity of $\lambda = 3$ would need to be $2$. That means that the eigenspace of $\lambda = 2$ would be of dimension $1$ and the eigenspace of $\lambda = 3$ would be of dimension $2$. In order to solve for the eigenvectors for each eigenvalue, we need to solve $Nul(A-\lambda I)$ for $\lambda=2, 3$.\\
    $\lambda = 2$: $(A-2I)=
    \begin{bmatrix}
    1 & a & b\\
    0 & 0 & c\\
    0 & 0 & 1
    \end{bmatrix}$. When solving for $(A-2I)=0$ we have,
    $\begin{bmatrix}
    1 & a & b & 0\\
    0 & 0 & c & 0\\
    0 & 0 & 1 & 0
    \end{bmatrix}$
    So for eigenvector $v = 
    \begin{bmatrix}
    v_1\\
    v_2\\
    v_3
    \end{bmatrix}$, the equation gives that $v_3 = 0, v_1 = -a*v_2-b*v_3 = -av_2$ and $v_2$ as a free variable. So the eigenspace would be
    $v_2\begin{bmatrix}
    -a\\
    1\\
    0
    \end{bmatrix}$
    \\
    $\lambda = 3$: $(A-3I)=
    \begin{bmatrix}
    0 & a & b\\
    0 & -1 & c\\
    0 & 0 & 0
    \end{bmatrix}$. Therefore, when solving $(A-3I)=0$ we have, 
    $\begin{bmatrix}
    0 & a & b & 0\\
    0 & -1 & c & 0\\
    0 & 0 & 0 & 0
    \end{bmatrix}$, for eigenvector $v = 
    \begin{bmatrix}
    v_1\\
    v_2\\
    v_3
    \end{bmatrix}$, we get $av_2+bv_3 = 0, v_2 = cv_3,$ and $v_1$ can be a free variable. Since this eigenspace must have dimension $2$ we will require that $c=0$ which allows $v_2=0$ and $v_3$ becomes a free variable. Then the eigenspace will become 
    $v_1 \begin{bmatrix}
    1\\
    0\\
    0
    \end{bmatrix} + v_3 \begin{bmatrix}
    0\\
    0\\
    1
    \end{bmatrix}$.\\
    Therefore, $c=0$ and $a, b$ can be any arbitrary real numbers.
    \item
    
    Let $A$ be an $n\times n$ matrix where it has $\lambda_1, \lambda_2, ..., \lambda_n$ distinct eigenvalues where $v_1, v_2, ..., v_n$ are the corresponding eigenvectors. Since there are $n$ distinct eigenvalues, and $n$ corresponding eigenvectors, the geometric multiplicity of each of the $n$ eigenvalues is $1$. Since each eigenvalue has geometric multiplicity of $1$, then the eigenspace of each of the eigenvalues is of dimension $1$. Therefore, if any vector space generated by one non-zero vector $v_1 (1\leq i\leq n)$, all vectors of $w_i$ are scalar multiple of $v_i$. This means that $w_i=\{u_i: u=kv_i\} k\in \text{field}$. Then any eigenvector $v$ of $A$ will be an element of any $1-$dim eigenspace $w_i$ generated by $v_i$ corresponding to eigenvalue $\lambda_i$. Then $v=k*v_i$ for some $v_i\in\{v_1, v_2,..., v_n\}$ and $k\in \text{Field}$. 
    
    \item
    
    So given that these vectors would be existing in $\mathbb{R}^n$, since the set of vectors $\{e_1, e_2, ..., e_n\}$ are the standard basis for $\mathbb{R}^n$, the Gram-Schmidt process would first convert the first vectors $\{u_1, u_2, ..., u_k\}$ to an orthonormal collection of vectors, then once it reaches the set $\{e_1, e_2, ..., e_n\}$ it would simplify some of those vectors to become orthonormal, but since there are, in total, more than $n$ vectors ($k+n>n$) for nonzero, positive $k$, some of the vectors would have to result as $0$ which would be skipped. Therefore, the Gram-Schmidt process would result in a set of $n$ orthonormal vectors which would also be a basis for $\mathbb{R}^n$.
    
    \item
    
    \begin{tabular}{|l|l|l|l|l|}
    \hline
            & -4 & -8 & -14 & -16 \\ \hline
    GS       &  $1*10^{-7}*0.3815$  & $0.5760$   &  $0.5000$   & $1.0000$    \\ \hline
    GS twice &  $1*10^{-11}*0.1140$  & $1*10^{-8}*0.6729$   &  $1*10^{-15}*0.2220$   &  $1.0000$   \\ \hline
    QR       &  $1*10^{-15}*0.6661$  & $1*10^{-15}*0.2282$  &  $1*10^{-15}*0.2220$   &  $1*10^{-15}*0.2220$   \\ \hline
    \end{tabular}
    
\end{enumerate}

\end{document}
