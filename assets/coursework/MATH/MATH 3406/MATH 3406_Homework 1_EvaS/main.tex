\documentclass{article}
\usepackage[utf8]{inputenc}
\usepackage[english]{babel}
\usepackage{amsthm}
\usepackage{amssymb}
\usepackage{mathcomp}
\usepackage{amsmath}
\usepackage{natbib}

\newtheorem{ishaan}{Theorem}[section]
\newtheorem{lemma}{Lemma}[section]
\renewcommand\qedsymbol{$\blacksquare$}

\title{Homework 1}
\author{Sean Eva}
\date{January 25, 2021}

\begin{document}

\maketitle

\begin{enumerate}
    \item Let $Ax = b$ be a linear system with real coefficients such that $A$ is an $m \times n$ matrix and $Ax = b$ has a unique solution.
    What can you say about $m, n$?
    Explain how to pick a subsystem of $n$ equations in $Ax = b$ such that the two systems have the same solutions.
    (Just present a correct algorithm; you don’t have to prove that the algorithm works).\\
    
    In this situation, $m$ is the number of equations that are present in the linear system and $n$ is the number of unknowns the equations have. For example, if $A$ has dimensions $3 \times 4$, then the system would contain $3$ equations in $4$ unknowns.\\
    In order to come up with a pair of systems that will produce the same solutions, you simply need to put the matrix A into Row Echelon Form, then take the equations whose rows contain the pivots. There should be n equations given that the system as a unique solution. Use exclusively these equations to form an $n\times n$ matrix that would have the same solution.\\
    
    \item Read the essay on floating point arithmetic in Appendix A (in the course notes to be found in Canvas) and write a paragraph (under 50 words) on why round-off errors are inevitable in the floating point arithmetic.\\
    
    In floating point arithmetic, the rewritten number can only have a certain degree of precision. If a number contains more information that extends beyond the degree of precision, that information will be lost. This means that round-off errors are inevitable because it will almost always lose information.\\
    
    \item Read the essay on row reduction with partial pivoting in Appendix D and fill details in the claim that the equation $-9999x_2 = -9998$ can be written as $-(.1) \times 10^5x_2 = -(.1) \times 10^5$.\\
    
    Through the use of rounded arithmetic, we can estimate the values of $-9999x_2 = -9998$ to be of base $10$, precision $3$, and exponent range $[-16, 16]$. Rounded arithmetic desires for numbers to be written in the form $\pm 0.d_1d_2d...d_t\times \beta^e$ for $d_1, d_2, ..., d_t$ being non-negative integers between $0$ and $9$ where $d_1\neq 0$ and $e$ is an integer between a defined exponent range. This allows for the original equation to be written as $-(0.1)\times 10^5 x_2 = -(0.1)\times 10^5$.\\
    
    \item The MATLAB command $hilb(n)$ calls the $n \times n$ Hilbert matrix that is famously illconditioned.
    Let’s play with it a bit.
    Find the smallest $n$ such that $inv(hilb(n))-invhilb(n)$ has an entry with absolute value at least $1$.
    Here $inv(hilb(n))$ is the numerical inverse of $hilb(n)$ and $invhilb(n)$ is the exact inverse.
    Discuss what this has to do with errors in solving numerically the linear system $Ax = b$ where $A = hilb(n)$ and $b$ is a column of the identity matrix.\\
    
    Given that there was a difference between $inv(hilb(n))$ and $invhilb(n)$, that means that there is an amount of error that occurs during the calculation process, depending on which inverse you find.\\
    
    \item Let A be a real matrix such that the linear system Ax = 0 has a nonzero complex solution. Show that it has a nonzero real solution.\\
    
    \begin{proof}
    Suppose for the sake of contradiction that the real matrix $A$ does not have real solutions to the linear system $Ax=0$. However, we know the system has a solution matrix $x$ that has a nonzero complex solution. If this is the only solution, that means that there must be a complex component of matrix $A$ in order for the product of complex numbers to result in 0. That is a contradiction with the definition of matrix $A$ in that the matrix is real.
    \end{proof}
    
\end{enumerate}

\end{document}
