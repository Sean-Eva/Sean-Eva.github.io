\documentclass{article}
\usepackage[utf8]{inputenc}
\usepackage[english]{babel}
\usepackage{amsthm}
\usepackage{amssymb}
\usepackage{mathcomp}
\usepackage{amsmath}
\usepackage{natbib}
\usepackage{array}
\usepackage{wrapfig}
\usepackage{multirow}
\usepackage{tabularx}

\newtheorem{ishaan}{Theorem}[section]
\newtheorem{lemma}{Lemma}[section]
\renewcommand\qedsymbol{$\blacksquare$}

\title{Homework 8}
\author{Sean Eva}
\date{April 2021}

\begin{document}

\maketitle

\begin{enumerate}
    \item 
    
    First it would be helpful to show that if $Bv=\lambda v$ then $BAv=\lambda Av$ for arbitrary eigenvalue $\lambda$ and corresponding eigenvector $v$.
    \begin{align*}
        BAv&=ABv\\
        &=A(Bv)\\
        &=A(\lambda v)\\
        &=\lambda Av.
    \end{align*} Then to show that B and A share an eigenvector. Since $AB=BA$, we know that $B^{-1}AB=A.$ Since we know that $BAv=\lambda Av$,
    \begin{align*}
        BAv&=\lambda Av\\
        [B-\lambda I]Av&=0.
    \end{align*} Therefore, $Av$ is also an eigenvector of $B$. We know that it will be a multiple of $v$ which means that it will also be an eigenvector of $A$.
    
    \item
    
    Since $P$ is an orthogonal projection of $\mathbb{R}^n$ onto $W$, we know that $Px=x, \forall x\in W$ and $py=0,\forall y\in W^\perp$. If we let $dim(W)=k$ for $0<k<n$, we can define $\{u_1, u_2, ..., u_k\}$ to be an orthonormal basis of $W$. We could then extend this basis to be a basis of $\mathbb{R}^n$ as $\{u_1, u_2,..., u_k, u_{k+1}, ..., u_n\}$ where $\{u_{k+1}, ..., u_n\}$ will form a basis for $W^\perp$. Therefore $Pu_i=u_i$ for $1\leq i\leq k$ and $Pu_j=0$ for $k+1\leq j\leq n$. Therefore, we could say that $\{u_1, u_2, ..., u_k\}$ are eigenvectors of $P$ corresponding to the eigenvalue $1$. Similarly, $\{u_{k+1}, ..., u_n\}$ are eigenvectors of $P$ corresponding to eigenvalue $0$. Since $Q$ is orthogonal, we know that the columns of $Q$ are orthonormal and we are given that $Q^T=Q^{-1}$. Let us define diagonal matrix $D$ to have the eigenvalues of $P$ as the elements of the diagonal where $a_{i,i}=\begin{cases}
    1 & 1\leq i\leq k\crcr
    0 & k+1\leq i\leq n
    \end{cases}$. Therefore, $PQ=QD, P=QDQ^{-1}=QDQ^T$. Thus, $Px=QDQ^Tx$.
    
    \item
    
    The maximum value of $n$ such that norm $<10^{-2}$ is when $n=69$. The Schur decomposition is quite robust as we can see in this problem, even when we calculate when the norm is $<10^{-2}$, the complexity of $A$ is well beyond a reasonable scope that can be achieved by hand. Therefore, it will help in creating simplified representations of a matrix like $A$ as a product of three matrices that can provide information on the matrix $A.$
    
    \item
    
    The largest $n$ for $B^n D B^{-n}$ is $n=9$. The largest $n$ for $Q^nDQ^{-1}$ is around $1.25*10^{17}$. This means that the second process of orthogonalizing the columns of $B$ is much more reliable and accurate to the production of the eigenvalues.
    
\end{enumerate}

\end{document}
