\documentclass[11pt]{article}

%%%%%%%%%%%%%%%%%%%%%%%%
%%%%%%%%%%%%%%%%%%%%%%%%
%%%%%%Packages
%%%%%%%%%%%%%%%%%%%%%%%%
%%%%%%%%%%%%%%%%%%%%%%%%

\usepackage{amsthm}
\usepackage{amsmath}
\usepackage{amssymb}
\usepackage[margin=1in]{geometry}
\usepackage{enumerate}
%\usepackage{hyperref}
%\usepackage{mathrsfs}
%\usepackage{color}
%\usepackage{bm}



%%%%%%%%%%%%%%%%%%%%%%%%
%%%%%%%%%%%%%%%%%%%%%%%%
%%%%%%amsthm settings
%%%%%%%%%%%%%%%%%%%%%%%%
%%%%%%%%%%%%%%%%%%%%%%%%

\theoremstyle{definition}
\newtheorem{problem}{Problem}
\newtheorem{claim}{Claim}
\newtheorem{definition}{Definition}

%%%%%%%%%%%%%%%%%%%%%%%%
%%%%%%%%%%%%%%%%%%%%%%%%
%%%%%%Custom commands: mathbb
%%%%%%%%%%%%%%%%%%%%%%%%
%%%%%%%%%%%%%%%%%%%%%%%%

\newcommand{\A}{\mathbb A}
\newcommand{\C}{\mathbb{C}}
\newcommand{\D}{\mathbb{D}}
\newcommand{\E}{\mathbb{E}}
\newcommand{\F}{\mathbb{F}}
\newcommand{\N}{\mathbb{N}}
\renewcommand{\P}{\mathbb{P}}
\newcommand{\R}{\mathbb{R}}
\newcommand{\X}{\mathbb{X}}
\newcommand{\Z}{\mathbb{Z}}
\newcommand{\Q}{\mathbb{Q}}

%%%%%%%%%%%%%%%%%%%%%%%%
%%%%%%%%%%%%%%%%%%%%%%%%
%%%%%%Custom commands: greek
%%%%%%%%%%%%%%%%%%%%%%%%
%%%%%%%%%%%%%%%%%%%%%%%%

\renewcommand{\a}{\alpha}
\renewcommand{\b}{\beta}
\newcommand{\g}{\gamma}
\renewcommand{\d}{\delta}
\newcommand{\e}{\epsilon}
\renewcommand{\l}{\lambda}

\begin{document}

\noindent {\bf MATH 4317 Homework 3}  \hfill Due Friday, 2/4/2022 at 11PM to Gradescope

\bigskip

\noindent Homework guidelines: 
\begin{itemize}
\item Each problem I assign, unless otherwise stated, is asking you to prove something. Give a full mathematical proof using only results from class or Wade.
\item Submit a PDF or JPG to gradescope. The grader has $\sim 250$ proofs to grade:  please make his job easier by submitting each problem on a different page. 
\item If you submit your homework in Latex, you get 2\% extra credit. 
\end{itemize}

\subsection*{Problems (5 total, 10 pts each)}

\begin{problem}
Prove, directly from the definition of convergence of a sequence, that $\frac{1}{n^2} \to 0$ as $n \to \infty$. You may use the fact that if $E \subset \Z$ is bounded from above, then $\sup E \in E$. {\bf You may not use the results of Wade Section 2.2, nor may you use the fact that $\frac1n \to 0$ as $n \to \infty$. }

\emph{Hint: Try to emulate the proof that $\frac1n \to 0$ as $n \to \infty$.}
\end{problem}

\begin{proof}
Let $\epsilon > 0$ and by using the Archimedean Principle we will choose $N\in \N$ such that $N > \sqrt{\frac{1}{\epsilon}}.$ Therefore, we have that if $n> N > \sqrt{\frac{1}{\epsilon}}$ implies that $n^2 > \frac{1}{\epsilon}$ implies that $\frac{1}{n^2}<\epsilon.$. Since $\frac{1}{n^2}$ is positive for all $n$ it follows that $|\frac{1}{n^2}|<\epsilon$ for all $n \geq N.$
\end{proof}

\newpage

\begin{problem}
Let $a, b \in \R, a < b$. Prove, directly from the definition of convergence of a sequence, that if $(x_n)$ is such that $a \leq x_n \leq b$ for all $n$ and $x_n \to x$ for some $x \in \R$, then $x \in [a,b]$. {\bf You may not use the results of Wade Section 2.2.
}\end{problem}

\begin{proof}
Let $\epsilon > 0$. We know from above that $a\leq x_n \leq b$ for all n. By the definition of convergence we know that for some $N\in \N$ and then for $n> N$ we have that $|x_n-x|<\epsilon.$ From this we can then write that $x<x_n+\epsilon \leq b+\epsilon$ and similarly $a-\epsilon \leq x_n-\epsilon < x$. Therefore, this implies that $a-\epsilon < x < b+\epsilon$. Since this holds for any $\epsilon >0$ we have that $a\leq x \leq b$ as desired.
\end{proof}

\newpage

For the remaining problems, you are allowed to use the results of Wade Section 2.2: 

\begin{problem}
Assume $x > 0$ and $(x_n)$ is a sequence of nonnegative numbers such that $x_n \to x$. Prove that $\sqrt{x_n} \to \sqrt{x}$. 

\emph{Hint: You may find it useful to use the following identity: 
\[
\sqrt a - \sqrt b = \frac{a - b}{\sqrt a + \sqrt b}
\].}
\end{problem}

\begin{proof}
Let $\epsilon > 0$, then by Archimedean Principle we will choose $N\in \N$ such that $N > \epsilon\sqrt{x}$. Since we are given that $|x_n-x|<\epsilon$ we also know that $|x_n-x|<\epsilon\sqrt{x}$. Then we have that $|\sqrt{x_n}-\sqrt{x}| = |\frac{x_n-x}{\sqrt{x_n}+\sqrt{x}}|$ and since the sequence is nonnegative numbers we know that $|\sqrt{x_n} + \sqrt{x}| = \sqrt{x_n} + \sqrt{x}$. Then it follows that $|\frac{x_n-x}{\sqrt{x_n} + \sqrt{x}}| = \frac{|x_n-x|}{\sqrt{x_n} + \sqrt{x}} < \frac{|x_n-x|}{\sqrt{x}} < \frac{\epsilon\sqrt{x}}{\sqrt{x}}=\epsilon$. Therefore for $n > N$ we have that $|\sqrt{x_n}-\sqrt{x}| < \epsilon$ then we know that $\sqrt{x_n}\rightarrow \sqrt{x}.$
\end{proof}

\newpage

\begin{problem}
Let $(x_n)$ be defined recursively by 
\[
x_{n + 1} = 2 + \sqrt{x_n -2} \,,
\]
where $x_1$ is assumed to be $> 3$. Note that the sequence $(x_n)$ is defined. 
Prove that if $x_n \to x$ for some $x \in \R$, then $x = 3$. 
\end{problem}

By theorem 2.8 we know that every convergent sequence is bounded, and since $x_n$ is convergent it is also bounded. I am first going to show that $x_n$ is bounded from below by $3.$
\begin{proof}
We are going to prove this statement by mathematical induction.\\
Base case: We are given that $x_1>3$ by definition so it is true for $n=1$.\\
Inductive step: Let's assume that $x_n>3$ we then want to show that $x_{n+1}>3$. Then $x_{n+1} = 2 + \sqrt{x_n-2} > 2 + \sqrt{3-2} = 2 + \sqrt{1} = 3.$ Therefore, $x_{n+1} > 3$ and further that $x_n$ is bounded from below by $3.$
\end{proof}
Given this we are going to prove the original problem statement.
\begin{proof}
Given that $x_n \rightarrow x$ for $x\in \R$, we can say that as $n \rightarrow \infty$ then we can say that $x_n \rightarrow x$ and $x_{n+1} \rightarrow x$. Then we can continue to say that,
\begin{align*}
    x &= 2 + \sqrt{x - 2}\\
    x - 2 &= \sqrt{x - 2}\\
    (x - 2)^2 &= x - 2\\
    x^2 - 4x + 4 &= x - 2\\
    x^2 - 5x + 6 &= 0\\
    (x - 2)(x - 3) &= 0\\
    x &= 2, 3.
\end{align*} But since we know that $x_n$ is bounded below by 3, we can eliminate the fact that $x = 2$ as it is not the biggest lower bound. So we can then determine that if $x_n \rightarrow x$ for some $x\in \R$, then $x = 3.$
\end{proof}

\newpage


\begin{problem}
Does the sequence
\[
x_n = \frac{(-1)^n n^3 + 2 n^2 + 1}{3 n^3 + (-1)^n n + 13} 
\]
converge? Prove your answer. 

{\it Hint: Consider the sequences $y_n = x_{2n}$ and $w_n = x_{2n + 1}$ for $n \in \N$. What can you say about $(y_n), (w_n)$?}
\end{problem}

\begin{proof}
In order to attempt to solve for this convergence we are going to multiply the top and bottom of this fraction by $\frac{1}{n^3}.$ Then we have that $\lim_{n\rightarrow \infty}\frac{(-1)^n n^3 + 2 n^2 + 1}{3 n^3 + (-1)^n n + 13} = \lim_{n\rightarrow \infty} \frac{(-1)^n + \frac{2}{n} + \frac{1}{n^3}}{3 + \frac{(-1)^n}{n^2} + \frac{13}{n^3}}$. By Theorem 2.12 (iv) we have that $\lim_{n\rightarrow \infty} \frac{(-1)^n + \frac{2}{n} + \frac{1}{n^3}}{3 + \frac{(-1)^n}{n^2} + \frac{13}{n^3}} = \frac{\lim_{n\rightarrow \infty} (-1)^n + \frac{2}{n} + \frac{1}{n^3}}{\lim_{n\rightarrow \infty}3 + \frac{(-1)^n}{n^2} + \frac{13}{n^3}}$. By simplifying this we know that $(-1)^n$ does not converge in the numerator and in the denominator we can conclude that $\lim_{n\rightarrow \infty}3 + \frac{(-1)^n}{n^2} + \frac{13}{n^3} = 3 + \lim(\frac{(-1)^n}{n^2}) + 0 = 3 + \lim((-1)^n\frac{1}{n^2}) = 3+ 0 = 3$. Therefore we are left with $\lim_{n\rightarrow \infty} \frac{(-1)^n}{3}$ and we know from example 2.3 that $(-1)^n$ does not converge so we know that the whole sequence does not converge.
\end{proof}

\end{document}
