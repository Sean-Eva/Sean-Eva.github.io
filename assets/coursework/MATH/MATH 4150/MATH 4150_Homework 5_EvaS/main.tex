\documentclass{article}

%%%%%%%%%%%%%%%%%%%%%%%%
%%%%%%%%%%%%%%%%%%%%%%%%
%%%%%%Packages
%%%%%%%%%%%%%%%%%%%%%%%%
%%%%%%%%%%%%%%%%%%%%%%%%

\usepackage{amsthm}
\usepackage{amsmath}
\usepackage{amssymb}
\usepackage[margin=1in]{geometry}
\usepackage{enumerate}



%%%%%%%%%%%%%%%%%%%%%%%%
%%%%%%%%%%%%%%%%%%%%%%%%
%%%%%%amsthm settings
%%%%%%%%%%%%%%%%%%%%%%%%
%%%%%%%%%%%%%%%%%%%%%%%%

\theoremstyle{definition}
\newtheorem{problem}{Problem}
\newtheorem{claim}{Claim}
\newtheorem{definition}{Definition}

%%%%%%%%%%%%%%%%%%%%%%%%
%%%%%%%%%%%%%%%%%%%%%%%%
%%%%%%Custom commands: mathbb
%%%%%%%%%%%%%%%%%%%%%%%%
%%%%%%%%%%%%%%%%%%%%%%%%

\newcommand{\A}{\mathbb A}
\newcommand{\C}{\mathbb{C}}
\newcommand{\D}{\mathbb{D}}
\newcommand{\E}{\mathbb{E}}
\newcommand{\F}{\mathbb{F}}
\newcommand{\N}{\mathbb{N}}
\renewcommand{\P}{\mathbb{P}}
\newcommand{\R}{\mathbb{R}}
\newcommand{\X}{\mathbb{X}}
\newcommand{\Z}{\mathbb{Z}}
\newcommand{\Q}{\mathbb{Q}}

%%%%%%%%%%%%%%%%%%%%%%%%
%%%%%%%%%%%%%%%%%%%%%%%%
%%%%%%Custom commands: greek
%%%%%%%%%%%%%%%%%%%%%%%%
%%%%%%%%%%%%%%%%%%%%%%%%

\renewcommand{\a}{\alpha}
\renewcommand{\b}{\beta}
\newcommand{\g}{\gamma}
\renewcommand{\d}{\delta}
\newcommand{\e}{\epsilon}

\title{Homework 5}
\author{Sean Eva}
\date{April 2022}

\begin{document}

\maketitle

\begin{enumerate}
    \item [1. ]
    
    \begin{proof}
    \begin{align*}
        10! + 1 &= 10 * 9 * 8 * 7 * 6 * 5 * 4 * 3 * 2 * 1 + 1 (\mod 11)\\
        10! + 1 &= (4 * 3)(2 * 6)(8 * 7)(9 * 5) * 10 + 1(\mod 11)\\
        10! + 1 &= (12)(12)(56)(45) * 10 +1 (\mod 11)\\
        10! + 1 &= (1)(1)(1)(1)*10+1(\mod11)\\
        10! + 1 &= 11(\mod 11)\\
        10! + 1 &= 0 (\mod 11).
    \end{align*}
    \end{proof}
    
     \item [17. ]
    
    \begin{proof}
    It will be useful to employ Wilson's Theorem, which states that if $p$ is a prime, then, $(p-1)! \equiv -1(\mod p).$ Also that $(p-1)! = (p-1)(p-2)(p-3)!$ allows for $(p-1)! \equiv (-2)(-1)(p-3)!(\mod p)$ which further implies that $2(p-3)! \equiv -1 (\mod p).$
    \end{proof}
    
     \item [41. ]
    
    \begin{proof}
    Given $p$ is a prime, then $1*2*...*(p-1)\equiv (p+1)(p+2)...(2p-1)(\mod p)$ each factor is prime to $p$. So $1 \equiv \frac{(p+1)(p+2)...(2p-1)}{1*2*...*(p-1)}(\mod p)$. Therefore, $2 \equiv \frac{(p+1)(p+2)...(2p-1)(2p)}{1*2*...*(p-1)}(\mod p)$ which means that $\binom{2p}{p}(\mod p)$
    \end{proof}
    
     \item [45. ]
    
    \begin{enumerate}
        \item 
        
        If $c < 26$ then $c$ cards are put into the deck above the card so it ends up in the $2c$ position and $2c < 52.$ So $b = 2c$, if $c \geq 26$ then the card is in the $c-26th$ place in the bottom half of the deck. In teh shuffle $c-26-1$ cards are put into the deck above the card so it ends up in the $b = (c-26+c-26-1)th$ place then $b = 2c - 53 \equiv 2c(\mod 53).$
        
        \item
        
        Since the shuffling is occuring in such a way that card at each shuffle chooses a different position and does not repeat the position until it goes over all the possible 51 positions and hence the required shuffle of number is $51 + 1 = 52.$
        
    \end{enumerate}
    
     \item [1. ]
    
    \begin{proof}
    For 91 to be pseudoprime base 3 would mean that it can be defined as $q$ and write $3^q \equiv 3 (\mod q)$ which is true as $3^91 \equiv 3 (\mod 91)$. However, we know that $91 = 7*13$ which means that it is composite. Therefore we know that $91$ is pseudoprime base 3.
    \end{proof}
    
     \item [9. ]
    
    \begin{proof}
    Since we know that $n$ is a pseudoprime to the bases $a$ and $b$ then we know that $a^n \equiv a (\mod n)$ and $b^n \equiv b (\mod n)$. So then we get,
    \begin{align*}
        a^n*b^n &= a*a*a*a*...*a*b*b*b*b*...*b\\
        a^n b^n &= (ab)^n\\
        a^nb^n &= a*b(\mod n)\\
        (ab)^n &= ab(\mod n).
    \end{align*} Therefore given that $n$ is pseudoprime to bases $a$ and $b$ we know then that $n$ is pseudoprime to base $ab.$
    \end{proof}
    
     \item [3. ]
    
    \begin{proof}
    Let $m > 2$ then $\phi(m)$ is even number. Also if $gcd(a, m) = 1$ if and only if $gcd(m-1, m) = 1$. So we arrange $c_1, c_2, ..., c_{\phi(m)}$ such that $c_{\phi(m)} = m - c_1, c_{\phi(m)-1} = m-c_2.$ So $c_1, c_2, ..., c_{\phi(m)/2}, (m-c_1), (m-c_2), ..., m - c_{\phi(m)/2}$ is the complete list of reduced residue system. So $c_1+ c_2 + ... + c_{\phi(m)} = \frac{\phi(m)}{2}*m \equiv 0 (\mod m).$ Thus $c_1 + c_2 + ... + c_{\phi(m)} \equiv 0 (\mod m)$ \end{proof}
    
     \item [6. ]
    
    \begin{proof}
    It will be important to notice that $\phi(10) = 4$ and that implies that $7^4 \equiv 1 (\mod 10)$. Then we get,
    \begin{align*}
        7^{999999} &\equiv 7^{3} * 1 (\mod 10)\\
        &\equiv 343 (\mod 10)\\
        &\equiv 3 (\mod 10).
    \end{align*}
    \end{proof}
    
     \item [14. ]
    
    \begin{proof}
    Consider $M_k = M / m_k = m_1m_2...m_{k-1}m_{k+1}...m_r$ for the above congruency, if $j\neq k$ then $(M_j, m_k) = 1.$ Therefore, $(M_k, m_k) = 1$. Now $M_k$ has an inverse $m_k$ we will denote $y_k$ which means that $M_ky_k \equiv 1 (\mod m_k)$. Therefore, the sum can be written as $x \equiv a_1M_1y_1 + a_2M_2y_2 + ... + a_rM_ry_r$. The integer $x$ is a simultaneous solution of the r congruences. And because $m_k | M_j$ whenever $j \neq k$, therefore, $M_j \equiv 0 (\mod m_k)$. Thus, in the sum of $x,$ all terms except the $kth$ term are congruent to $0(\mod m_k).$ And because $M_ky_k \equiv 1 (\mod m_k)$. Put the values in the equation to get $x \equiv a_1M_1^{\phi(m_1)} + ... + a_rM_r^{\phi(m_r)} (\mod M)$ as desired.
    \end{proof}
    
     \item [5. ]
    
    \begin{proof}
    Given that $\phi(n)$ is multiplicative. Let $n = 2^ap_1^bp_2^c...p_k^\alpha$ where $p_i$ are distinct odd primesm the $b, c, ..., \alpha \geq 1$ and $a \geq 0.$ Then, $\phi(n) = \phi(2^a)\phi(p_1^b)...\phi(p_k^\alpha)$. We find all $n$ such that $\phi(n) = 6$. If $k \geq 2$, then since $\phi(p_i^{e_i})$ is even, $\phi(n)$ is divisible by $4$, so cannot be equal to $6.$ If $k = 0$ we cannot have $\phi(n) = 6$. We conclude that $k = 1$. Thus $n$ must have the shape $2^ap^e$, where $a\geq 0$ and $p$ is an odd prime. But $\phi(p^e) = p^{e-1}(p-1).$ It follows that $p \leq 7$. If $p = 7$, then $p-1 = 6$, so we must have $e = 1$ and $\phi(2^a) = 1.$ This gives the solutions $n = 7$ and $14.$ We cannot have $p = 5$ because $4 | \phi(5^e)$. Let $p = 3.$ If $e \geq 3,$ then $\phi(e^e) \geq (3^2)(2)$. So we are left with the possibilities that $e = 1, 2.$ If $e = 1,$ then $\phi(n) = \phi(2^a)(2).$ This is cannot be $6.$ Finally if $e = 2$, then $\phi(3^2) = 6.$ So to have that $\phi(2^a3^2)=6,$ we need $\phi(2^a) = 1$ which gives us that $n = 9, 18.$ Therefore, all the solutions to $\phi(n) = 6$ are $n = 7, 9, 18.$
    \end{proof}
    
     \item [11. ]
    
    \begin{proof}
    Consider that $3$ does not divide $n$. Then $\phi(3n) = \phi(3)\phi(n) = 2\phi(n)$ which implies that $\phi(3n) \neq 3\phi(n).$ Alternatively, consider that $3|n$ then let $n = 3^k*m$ where $m$ is not divisible by $3$, and $k\geq 1.$ Then $\phi(n) = \phi(3^km) = 2*3^{k-1}\phi(m)$; also, $3n = 3^{k+1}m,$ so $\phi(3n) = 2*3^k\phi(m) = 3\phi(n)$. Therefore, the only numbers that the statement $3\phi(n) = \phi(3n)$ is true is for $n$ that are divisible by $3.$
    \end{proof}
    
     \item [36. ]
    
    \begin{proof}
    Consider positive integers $m$ and $n$. Soncider the function $f$ such that $f(n) = \frac{\phi(n)}{n}$ and $f(m) = \frac{\phi(m)}{m}$. Therefore, we get that $f(mn) = \frac{\phi(mn)}{mn}$ or, $f(mn) = \frac{mn\Pi(1-\frac{1}{p_i})\Pi(1-\frac{1}{q_i})}{mn} = \frac{m\Pi(1-\frac{1}{p_i})}{m}\frac{n\Pi(1-\frac{1}{q_i}}{n} = \frac{\phi(m)}{m}\frac{\phi(n)}{n} = f(m)f(n).$ Therefore, the considered function is completely multiplicative.
    \end{proof}
    
     \item [4. ]
    
    \begin{proof}
    We will show first that $\sigma(n)$ is odd if $n$ is a power of $2$. Suppose that $n = 2^\alpha$, then $\sigma(2^\alpha) = \sum_{d|2^\alpha}d = 1 + 2 + 2^2 + ... + 2^\alpha = \frac{2^{\alpha + 1} - 1}{2-1} = 2^{\alpha + 1}-1$, and $\sigma(2^\alpha) = 2^{\alpha+1}-1$ is odd for all integers $\alpha \geq 0.$ Next suppose that $p$ is an off prime and that $\alpha$ is a positive integer, then $\sigma(p^\alpha) = 1 + p + p^2 + ... + p^\alpha = \frac{p^{\alpha + 1} - 1}{p -1}$, and $\sigma(p^\alpha)$ is odd if and only if the sum contains an odd number of terms, that is, if and only if $\alpha$ is an even integer. From the fundamental theorem of arithmetic, we see that $\sigma(n)$ is odd if and only if in the prime power decomposition of $n$ every odd prime occurs to an even power, that is, if and only if $n$ is a perfect square or $n$ is $2$ times a perfect square.
    \end{proof}
    
\end{enumerate}

\end{document}
