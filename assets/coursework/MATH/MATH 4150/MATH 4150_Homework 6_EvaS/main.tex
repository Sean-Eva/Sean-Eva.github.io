\documentclass{article}

%%%%%%%%%%%%%%%%%%%%%%%%
%%%%%%%%%%%%%%%%%%%%%%%%
%%%%%%Packages
%%%%%%%%%%%%%%%%%%%%%%%%
%%%%%%%%%%%%%%%%%%%%%%%%

\usepackage{amsthm}
\usepackage{amsmath}
\usepackage{amssymb}
\usepackage[margin=1in]{geometry}
\usepackage{enumerate}



%%%%%%%%%%%%%%%%%%%%%%%%
%%%%%%%%%%%%%%%%%%%%%%%%
%%%%%%amsthm settings
%%%%%%%%%%%%%%%%%%%%%%%%
%%%%%%%%%%%%%%%%%%%%%%%%

\theoremstyle{definition}
\newtheorem{problem}{Problem}
\newtheorem{claim}{Claim}
\newtheorem{definition}{Definition}

%%%%%%%%%%%%%%%%%%%%%%%%
%%%%%%%%%%%%%%%%%%%%%%%%
%%%%%%Custom commands: mathbb
%%%%%%%%%%%%%%%%%%%%%%%%
%%%%%%%%%%%%%%%%%%%%%%%%

\newcommand{\A}{\mathbb A}
\newcommand{\C}{\mathbb{C}}
\newcommand{\D}{\mathbb{D}}
\newcommand{\E}{\mathbb{E}}
\newcommand{\F}{\mathbb{F}}
\newcommand{\N}{\mathbb{N}}
\renewcommand{\P}{\mathbb{P}}
\newcommand{\R}{\mathbb{R}}
\newcommand{\X}{\mathbb{X}}
\newcommand{\Z}{\mathbb{Z}}
\newcommand{\Q}{\mathbb{Q}}

%%%%%%%%%%%%%%%%%%%%%%%%
%%%%%%%%%%%%%%%%%%%%%%%%
%%%%%%Custom commands: greek
%%%%%%%%%%%%%%%%%%%%%%%%
%%%%%%%%%%%%%%%%%%%%%%%%

\renewcommand{\a}{\alpha}
\renewcommand{\b}{\beta}
\newcommand{\g}{\gamma}
\renewcommand{\d}{\delta}
\newcommand{\e}{\epsilon}

\title{Homework 6}
\author{Sean Eva}
\date{April 2022}

\begin{document}

\maketitle

\begin{enumerate}
    \item [1. ]
    
    \begin{enumerate}
        \item 
        
        $\mu(12) = \mu(2*2*3) = 0$
        
        \item
        
        $\mu(15) = \mu(3*5) = (-1)^2 = 1$
        
        \item
        
        $\mu(30) = \mu(2*3*5) = (-1)^3 = -1$
        
        \item
        
        $\mu(50) = \mu(2*5*5) = 0$
        
        \item
        
        $\mu(1001) = \mu(7*11*13) = (-1)^3 = -1$
        
        \item
        
        $\mu(2*3*5*7*11*13) = (-1)^6 = 1$
        
        \item
        
        $\mu(10!) = \mu(10*9*8*7*6*5*4*3*2) = \mu(2*5*9*8*7*6*5*4*3*2) = 0$
        
    \end{enumerate}
    
    \item [11. ]
    
    \begin{proof}
    Consider two nonnegative integers $n = 36k + 8$ and $n + 1 = 36k + 9$. $n$ is divisible by $4$ which implies that it has a square of $2$ in the set of its factors and the second number, $n + 1$ is divisible by $9$ which implies that it has a square of $3$ in its factors. Since both of these numbers contain squares in their prime factorizations, then their Mobius Function evaluations are $0.$ This $\mu(n) + \mu(n + 1) = \mu(36k + 8) + \mu(36k + 9) = 0 + 0 = 0.$ Therefore, there are infinitely many consecutive positive integers such that their summation of their Mobius Function evaluations are 0.
    \end{proof}
    
    \item [15. ]
    
    \begin{proof}
    Consider an identity function $h(n) = n$. Then if $n$ is a positive integer, then $n = \sum\phi(d)$. Therefore, $h(n) = n = \sum\phi(d)$. Now using Mobius inversion formula we get $\phi(n) = \sum\mu(d)h(n/d) = \sum\mu(d)(n/d) = n\sum\mu(d)d.$ Therefore, if $n$ is a positive integer, then $\phi(n) = n\sum\mu(d)/d.$
    \end{proof}
    
    \item [17. ]
    
    \begin{proof}
    Consider a multiplicative function $f$ with $f(1) = 1$. If $F$ is multiplicative, then $f$ is also multiplicative. Thus, both $f$ and $\mu$ are also multiplicative. Now. as $f$ and $\mu$ are multiplicative, then their product $\mu f$ is also multiplicative. Similarly, the summation $\sum\mu(d)f(d)$ is also multiplicative. Therefore, $\sum\mu(d)f(d) = \mu(p^a)f(p^a) + \mu(p^{a-1})f(p^{a-1}) + ... + \mu(p)f(p) + \mu(1)f(1).$ According to the definition, for exponents greater than $1$, the value of $\mu(p^i) = 0$. Therefore, we can simplify this to be $\sum\mu(d)f(d) = \mu(p)f(p) + \mu(1)f(1) = 1 - f(p)$. Now, as $n = p_1^{a_1}p_2^{a_2}...p_k^{a_k}$ we get that $\sum\mu(d)f(d) = (1-f(p_1))(1-f(p_2))...(1-f(p_k)).$
    \end{proof}
    
    \item [23. ]
    
    
    \begin{proof}
    Consider the identity given by $\sum\mu(d)f(d) = (1-f(p_1))(1-f(p_2))...(1-f(p_k))$ where $f$ is a multiplicative function with $f(1) = 1$. And $n = p_1^{a_1}p_2^{a_2}...p_k^{a_k}$ as a prime factorization. We are able to write $\sum\mu^2(d)$ in the form of the identity stated above because it can be considered $f(d) = \mu(d)$ and also $\mu(1)=1.$ Therefore, $\sum\mu^2(d) = (1-\mu(p_1))(1-\mu(p_2))...(1-\mu(p_k))$ for primes $p_i.$ Since $\mu(p) = -1$ for prime $p$ we know that $\sum\mu^2(d) = (1-(-1))(1-(-1))...(1-(-1)) = 2*2*...*2 = 2^k.$ Therefore, we know that $\sum\mu^2(d) = 2^k$ where $k$ is the number of distinct prime factors of $d.$
    \end{proof}
    
    \item [4. ]
    
    NPWJE APNSP QESW
    
    \item [38. ]
    
    If we let $p_1p_2...p_m$ and $q_1q_2...q_m$ be two different plain text streams. If $k_1k_2...k_m$ be the keystream used to encrypt the two plain texts $E_{k_i}(p_i) = k_i + p_i(\mod 2)$ and $E_{k_i}(q_i) = k_i + q_i(\mod 2)$ and the corresponding ciphertext streams are $E_{k_i}(p_i) + E_{k_i}(q_i) = k_i + p_i + k_i + q_i(\mod 2) = 2k_i + p_i + q_i (\mod 2) = p_i + q_i(\mod 2).$ It can then also be shown that if someone can encrypt a bit string and have access to the resulting cipher string, the key string can be found. If we have a key of $0$ then the cipher text would be the same as the plain text and we would know that the key is $0$. Similarly, if the key is $1$ then we would know this because the cipher text would be different from the plain text.
    
    \item [1. ]
    
    $p = 97, q = 151$
    
    \item [3. ]
    
    It is known that $P \leq n$. If $(P, n) \neq 1,$ then there must be greatest common divisors of $P$ and $n$ must be one of factors of $n$ that is $p$ or $q$. $(P, n) = p$ or $(P, n) = q$. Now, using the Euclidean algorithm to find the greatest common factors of $(P, n)$, the Euclidean Algorithm will give us one of the factors of $n$ and divide $n$ by this calculated factor to get another factor. Therefore, if plaintext $P$ is not relatively prime to the enciphering modulus, then the cryptanalyst can factor $n.$
    
    \item [4. ]
    
    For any integer $n$, there are $n$ integers up to and including $n.$ Now consider the given $n = pq$. Therefore, the following are the integers that are no relatively prime to $n:$ $p, 2p, 3p, ..., qp; q, 2q, 3q, ..., (p-1)q.$ Therefore, there are $q + p - 1$ integers up to $n$ that are not relatively prime to $n.$ This is also the number of ways the interested event is expected to occur. Therefore, the required probability is $\frac{q+p-1}{n} = \frac{q}{n} + \frac{p}{n}-\frac{1}{n} = \frac{q}{pq} + \frac{p}{pq} - \frac{1}{pq} = \frac{1}{p} + \frac{1}{q} - \frac{1}{pq}.$
    
    \item [12. ]
    
    Fermat's Factorization method hints that an odd number can be written as a difference of two squares that when factored as $a^2-b^2 = (a+b)(a-b)$. If the primes are close together, then $b$ will be a small number that will be easily found even under guess and check and even faster using a computer based algorithm.
    
    \item [13. ]
    
    Consider a plaintext $P.$ Now consider two exponents $e_1, e_2.$ If we have $(e_1, e_2) = a$ then there are some $x, y$ such that $a = xe_1 + ye_2.$ The encryption of the first part is given as $C_1 = p^{e_1} (\mod n)$ and the second part is $C_2 = P^{e_2}(\mod n)$ where $C_1, C_2$ are the cipher texts. Now, since we now know $C_1, C_2, e_1, e_2$ we are able to easily compute $x, y$ as $C_1^x, C_2^y = P^{e_1x}P^{e_2y} = P^{e_1x+e_2y} = P^a (\mod n)$. If $a$ is relatively small, then it may not be difficult to computer the $a^th$ root of $P^a$ and thereby to recover $P.$
    
    \item [14. ]
    
    Let us say the three modules are pairwise, relatively prime. We can use the Chinese remainder theorem to solve the system of congruences and give us a least non-negative integer $x = p^3 (\mod(n_1, n_2, n_3).$ By construction $p <x_i$ for $i = 1, 2, 3,$ we will have $p^3 < x_1, x_2, x_3.$ It is guaranteed by the theorem that $x$ must be a perfect cube whose cube root is easily computable. And this will be a plaintext $= p$.
    
    \item [12. ]
    
    The objective is to show that if the integers $'a'$ and $'b'$ are relatively prime and $(\text{ord}_n a, \text{ord}_n b) = 1$ then $\text{ord}_n ab = \text{ord}_na\text{ord}_nb.$ As the integers $a$ and $b$ are prime integers, therefore $(a, n) = 1, (b, n) = 1$ which implies that $(\text{ord}_na, \text{ord}_n) = 1$. Now consider that $\text{ord}_na = k_1$ and $\text{ord}_nb = k_2$. Therefore, $a^{k_1}\equiv 1 (\mod n)$ and $a^{k_2}\equiv 1 (\mod n)$ implies that $a^{k_1k_2}\equiv 1 (\mod n), b^{k_1k_2}\equiv 1(\mod n)$. Now, if we multiply both of these equations we get that $(ab)^{k_1k_2} \equiv 1 (\mod n)$ which means that $\text{ord}_nab = k_1k_2.$ Therefore, it is clear that from the above that $\text{ord}ab = \text{ord}a\text{ord}b$ as desired. 
    
    \item [16. ]
    
    From Euler's Theorem we have that for a positive integer $a$ relatively prime to another integer $n$, $a^{\phi(n)}\equiv 1 (\mod n)$. Therefore, for a positive integer $m$ a prime number we can write that, $a^{m-1}\equiv 1(\mod m).$ We also know that $\text{ord}_ma|\phi(m).$ Therefore, if $\text{ord}_m a = m-1$ then it must divide $\phi(m).$ We also know that $\phi(m) \leq m - 1$. Therefore, $\phi(m) = m-1$, which implies finally that the positive integer $m$ must be a prime.
    
    \item [12. ]
    
    Let $p$ be a prime and let $r$ be a primitive root of $p.$ Then the inverse $r^{-1} = r^{p-2}$ is a primitive root as well. Thus, we can group the primitive roots in pairs of mutually inverse roots whenever $r$ and $r^{-1}$ are different from each other. If we investigate when $r$ and $r^{-1}$ can coincide we get that $r\equiv r^{-1}(\mod p)$ and $r^2 \equiv 1 (\mod p)$ implies that $r\equiv \pm 1(\mod p)$ which are not primitive roots if $p > 3$. So for $p>3$, the primitive roots group in pairs of mutually inverse primitive roots and their total product is congruent to $1 (\mod p).$ If $p = 2$ there is only one primitive root, 1. So the least positive residue of the product of all primitive roots is again $1 (\mod p).$ If $p = 3$ the only primitive root is $-1 \equiv 2.$ In this case, the least positive residue of the product of all primitive roots equal 2.
    
    \item [16. ]
    
    $\mathbb{F}_p^x = mn-3$ are elements in the field of $p$ elements $- \Z/(p-1)\Z = \Z/2q\Z$. The mapping $x \rightarrow 2x$ is a homomorphism which implies that $\Z/2q\Z \rightarrow \Z/q\Z$ is also a homomorphism. The kernel is $q^\Z/2q\Z = \Z/2\Z$ which has order 2. So, the image is a subgroup of index 2, and so has order $q$, and so s generated by any non-zero element. In multiplicative language, squares in $\mathbb{F}^x_p$ form a subgroup of order $q$
     generated by any element whose square is not 1. Each $a, 1 < a < p-1$ is such an element. For such $a, \text{ord}_pa^2 = q.$ As $\text{ord}_p(p-1) = 2, p-1$ is not a square. So, $(p-1)a^2\not\in (\#_p^x)^2$ for some $1 < a < p-1$. But its square is $a^4$ which also generates $a^2$. So $<(p-1)a^2>$ must be all of $\#_p^x,$ so $(p-1)a^2 = p-a^2$ is primitive.
\end{enumerate}

\end{document}
