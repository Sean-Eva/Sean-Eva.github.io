\documentclass[11pt]{article}

%%% Useful packages
\usepackage{amsfonts}
\usepackage{amssymb}
\usepackage{amsmath}
\usepackage{amsthm}

%%% Sets page margins, overriding 11pt article defaults
\setlength{\topmargin}{0 in}
\setlength{\headheight}{0 in}
\setlength{\headsep}{0 in}
\setlength{\oddsidemargin}{0in}
\setlength{\textwidth}{6.5 in}
\setlength{\textheight}{9 in}

%%% Useful for layout
\newcommand{\VF}{\vspace*{\fill}}
\newcommand{\HF}{\hspace*{\fill}}

%%% shorthand for sets
\newcommand{\R}{\mathbb{R}}
\newcommand{\Z}{\mathbb{Z}}
\newcommand{\N}{\mathbb{N}}
\newcommand{\Q}{\mathbb{Q}}

\DeclareMathOperator{\Ker}{Ker}

\begin{document}

{\noindent\Large\textbf{Homework 3 due Wed, Sept~22nd by 11am in Gradescope}}

\vspace{.25in}

{\large
\noindent
\textbf{Name: Sean Eva} \smallskip \\
\textbf{GTID: 903466156} \smallskip \\
%%% List anyone with whom you discussed the problems
\textbf{Collaborators:} \smallskip \\
%%% List all resources used OTHER than the textbook, lecture notes, 
%%% and previous homework assignments.
\textbf{Outside resources:} \smallskip
}

\pagebreak 

%%% Edit the directions below to remove anything which is not part of
%%% your solution.  A blank line starts a new paragraph.  '\\' is a 
%%% line break.   The default latex formatting is usually acceptable,
%%% so don't spend a significant amount of time making your solutions
%%% look pretty.  Do spellcheck.  PROOFREAD, more than once!

INSERT a ``pagebreak'' command between each problem (integer numbers).
Problem subparts (letter numbered) can be on the same page.

REMOVE all comments (within ``textit\{\}'' commands) before submitting
solutions.

DO NOT include any identifying information (name, GTID) except on the
first/cover page.


\begin{enumerate}

\item Let $G$ be a group and $N \triangleleft G$.  Suppose $G$ is cyclic.
\begin{enumerate}
\item Prove that $G/N$ is cyclic directly (i.e. from the definition).

\begin{proof}
Let $G$ be a group and $N \triangleleft G$. Suppose that $G$ is cyclic. Since $N$ is normal to $G$ and $G$ is cyclic, we know that $G$ is abelian. Consider $G/N = \{[a]|a\in G\}=\{Na|a\in G\}$ under the relation $ba^{-1}\in N$ and is a group under the operation $[a][b]=[ab]$. Since $G$ is cyclic we can write any element $a\in G$ as $g^i=a\in G$ for some $i\in \Z$. Therefore, we can write $Na = Ng^i \forall a\in G$. Similarly, we then know that $Na = Ng^i = (Ng)^i$ which implies that $G/N$ is cyclic and is generated by $Ng.$
\end{proof}

\item Prove that $G/N$ is cyclic using a homomorphism.

\begin{proof}

\end{proof}

\end{enumerate}
\pagebreak

\item Let $G$ be a group and $N \triangleleft G$. 
\begin{enumerate}
\item Suppose $G/N$ is abelian.  Prove $ab a^{-1} b^{-1} \in N$ for all $a,b \in G$.\\

\begin{proof}
Let $G$ be a group and $N \triangleleft G$, and that $G/N$ is abelian. Consider $a, b\in G$, then we have that $(aN)(bN) = (bN)(aN).$ Therefore, $Nab=Nba$, $Naba^{-1}b^{-1}=N$ which then implies that $aba^{-1}b^{-1}\in N.$
\end{proof}

\item Suppose $ab a^{-1} b^{-1} \in N$ for all $a,b \in G$.  Prove $G/N$ is abelian.\\

\begin{proof}
Let $G$ be a group and $N\triangleleft G$. Suppose that $aba^{-1}b^{-1}\in N$ for all $a, b\in G$. In order to show that $G/N$ is abelian, we need to show that $Nab=Nba$ for all $a,b\in G$. Then $(Nb)(Na)=Nba=(eba)N = (Ne)(Nba) = N(Nba) = (Naba^{-1}b^{-1})(Nba) = (Naba^{-1}b^{-1}ba) = Nab = (Na)(Nb)$ which implies that $G/N$ is abelian since $(Nb)(Na) = (Na)(Nb).$
\end{proof}

\end{enumerate}
\pagebreak

\item Let $G$ be a cyclic group of order $n$.  Prove that $G$ has
$\phi(n)$ distinct generators (where $\phi(n)$ is the Euler $\phi$-function).  
Specify their form explicitly.\\

\begin{proof}
Let $G$ be a cyclic group of order $n$. We will prove that the generators for $G$ will be of the form $\{g^s|0\leq s < n, gcd(s,n)=1\}$. In order for this to be a generator, the order of $g^s$ be equal to $n$. Let us say that the order of $g^s$ is equal to $k$ where $0<k\leq n$. Because of Lagrange's Theorem, we know that $k$ divides $n$, so we now need to show that $n$ divides $k.$ From Euclid's lemma, we know that we can rewrite $k= qn + r$ for $qr\in\N$ where $0\leq r < n$. Then, $e=(g^s)^k=(g^s)^{qn+r}=(g^s)^{qn}(g^s)^r=(g^s)^r=g^{sr}.$ If the order of $g$ is $n$, then we know that $n|sr$, but since the $\gcd (s,n)=1$, we know that $n|r$. This would then mean that $n\leq r$ or that $r=0$. Because $0\leq r< n$ we know that $r=0$ and that $k=qn$, so then $n|k$ and therefore, we know that $k=n$ since $0< k\leq n$. Therefore, $g^s$ is a generator of $G$ for $\gcd (s,n)=1$ which means that the group $G$ has $\phi (n)$ generators.
\end{proof}

\pagebreak

\item Let $G$ be a finite group of \textbf{even} order with identity $e$.  
Prove that there must be an element $a \in G$ with $a \neq e$ and $a^2 = e$.\\

\begin{proof}
Let $G$ be a finite group of even order with identity $e$. Suppose that $g\in G$ such that $g^2\neq e$ which would mean that $g\neq g^{-1}$, if we counted these pairs of $g,g^{-1}$, and we then have the identity elements $e$. This would mean that we have an odd number of elements. This would then mean that we have one more element $a\in G$ that doesn't have a pairing implying that $a=a^{-1}$ and $a^2=e.$
\end{proof}

\pagebreak
\item Assuming Problem~\ref{cancel} is true, prove that 
$U_n =  \{ [a] \in \Z_n \mid (a,n) = 1\}$ is a group under the 
product $[a][b] = [ab]$.\\

\begin{proof}
In order to prove that $U_n$ is congruent under multiplication modulo $n$ we need to show that it is nonempty, contains an identity, contains inverses, is closed under the operation, and is associative. However, given that problem 7 is true, we only need to prove that the set $U_n$ is closed under the operation and that the operation is associative. Luckily, multiplication modulo $n$ is associative, so then we know that the operation of $U_n$ is associative. Consider $a,b\in U_n$, that means that $\gcd(a, n)=1$ and $\gcd (b,n)=1$ which implies that $\gcd(ab,n)=1$ which therefore means that $ab\in U_n$ and the operation is closed. Therefore, $U_n$ is a group. 
\end{proof}

\pagebreak
\item Disprove Problem~\ref{cancel} if $G$ is an infinite set with the
same properties. \\

Consider $G$ is the set of natural numbers under addition. It would be true that $ax = ay$ forces $x = y$ and $ua = wa$ forces $u = w$ for every $a, x, y, u, w \in G$. However, there is no identity element for the natural numbers under the operation of addition and therefore, $G$ is not a group.

\pagebreak
\item 
Let $G$ be a finite nonempty set closed under an associative 
operation such that $ax = ay$ forces $x = y$ and $ua = wa$ forces 
$u = w$ for every $a, x, y, u, w \in G$.  
Prove that $G$ is a group.  \label{cancel}\\




\end{enumerate}

\end{document}

