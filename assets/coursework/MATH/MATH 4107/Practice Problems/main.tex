\documentclass[11pt]{article}

%%% Useful packages
\usepackage{amsfonts}
\usepackage{amssymb}
\usepackage{amsmath}
\usepackage{amsthm}

%%% Sets page margins, overriding 11pt article defaults
\setlength{\topmargin}{0 in}
\setlength{\headheight}{0 in}
\setlength{\headsep}{0 in}
\setlength{\oddsidemargin}{0in}
\setlength{\textwidth}{6.5 in}
\setlength{\textheight}{9 in}

%%% Useful for layout
\newcommand{\VF}{\vspace*{\fill}}
\newcommand{\HF}{\hspace*{\fill}}

%%% shorthand for sets
\newcommand{\R}{\mathbb{R}}
\newcommand{\Z}{\mathbb{Z}}
\newcommand{\N}{\mathbb{N}}
\newcommand{\Q}{\mathbb{Q}}

\DeclareMathOperator{\Ker}{Ker}

\title{Practice Problems}
\author{seva6 }
\date{November 2021}

\begin{document}

\maketitle

\section{Chapter 4 Section 1}

\subsection{Easy Problems}

\begin{enumerate}
    \item 
    
    $1, 5, 7, 11, 13, 17, 23$
    
    \item
    
    Let $R$ be a ring such that $R$ is a field. That is to say that $R$ is commutative and a division ring. Given that $R$ is a division ring, it is true that $R$ has a unit $1$ and that for every $a\in R, a\neq 0$ there is a corresponding $a^{-1}\in R$ such that $a*a^{-1}=a^{-1}*a=1$, the unit in $R$. This means that $R$ has a multiplicative identity. Let $a,b\in R$ such that $a*b=0$ where $a\neq 0.$ This then implies, by the definition of a ring, that $a^{-1}$ exists. Then we can say that $a^{-1}(ab)=a^{-1}(0)=0=(1)b=(a^{-1}a)b,$ which implies that $b=0.$ Similarly, if $b\neq 0$ then $a=0.$ Therefore, this implies that whenever $ab=0,$ either $a=0$ or $b=0$ which means that $R$ is an integral domain.
    
    \item
    
    In order for an element $a\in \Z_n$ to have an inverse in $\Z_n$ it must be that $\gcd(a,n)=1$. In order for $\Z_n$ be a field, each element $a\in \Z_n$ must have an inverse and it must be commutative. If $n$ is not prime, we know that there exists $a<n$ such that $\gcd(a,n)\neq 1$. Therefore $a$ would not have an inverse in $\Z_n$. Therefore, in order for each element to have an inverse in $\Z_n$ then $n$ must be prime. Additionally, $\Z_n$ is commutative by virtue that $\Z_n$ is commutative by definition.
    
    \item
    
    In example information.
    
    \item
    
    In example information.
    
    \item
    
    Just multiply $3$ generic matrices.
    
    \item
    
    Matrix multiplication.
    
    \item
    
    $a,d$ can be anything but $b=c=0.$
    
    \item
    
    
    
    \item
    
\end{enumerate}

\subsection{Medium Problems}

Not worth the busy work.

\subsection{Hard Problems}

Not worth the busy work.

\section{Chapter 4 Section 2}

\begin{enumerate}
    \item 
    
    Let $R$ be a ring. That is to say that $R$ is an abelian group under addition. Then for $n,m\in \Z$ and $a,b\in R$ then $(na)(mb)=(a+a+a+...)(b+b+b+...)$ $n$ and $m$ times respectively which is then $(ab+ab+ab+...)+(ab+ab+ab...)$ where each parenthesis has $n$ times and this is repeated $m$ times. This then is to say that we have $ab+ab+ab...$ $nm$ times in total. Therefore, $(na)(mb)=(nm)(ab).$
    
    \item
    
    Let $R$ be an integral domain. That is to say that $R$ is commutative and it is true that for $a,b\in R$ that $ab=0$ then $a=0$ or $b=0.$ Then consider for $a,b,c\in R$ such that $a\neq 0$ and $ab=ac$. Then we can say that $ab-ac= a(b-c) = 0$ since $a\neq 0$ we know that $b-c=0$ which implies that $b=c$.
    
    \item
    
    Homework 5 problem 5.
    
    \item
    
    
    
    \item
    
    
    
    \item
\end{enumerate}

\section{Chapter 4 Section 3}

\subsection{Easy Problems}



\subsection{Medium Problems}



\subsection{Hard Problems}



\section{Chapter 4 Section 4}

\subsection{Easy Problems}



\subsection{Medium Problems}



\subsection{Hard Problems}





\end{document}
