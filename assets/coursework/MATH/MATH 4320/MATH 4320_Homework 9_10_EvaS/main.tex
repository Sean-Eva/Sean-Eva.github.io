\documentclass{article}


%%%%%%%%%%%%%%%%%%%%%%%%
%%%%%%%%%%%%%%%%%%%%%%%%
%%%%%%Packages
%%%%%%%%%%%%%%%%%%%%%%%%
%%%%%%%%%%%%%%%%%%%%%%%%

\usepackage{amsthm}
\usepackage{amsmath}
\usepackage{amssymb}
\usepackage[margin=1in]{geometry}
\usepackage{enumerate}
\usepackage{color}
\usepackage{graphicx}



%%%%%%%%%%%%%%%%%%%%%%%%
%%%%%%%%%%%%%%%%%%%%%%%%
%%%%%%amsthm settings
%%%%%%%%%%%%%%%%%%%%%%%%
%%%%%%%%%%%%%%%%%%%%%%%%

\theoremstyle{definition}
\newtheorem{problem}{Theorem}
\newtheorem{claim}{Claim}
\newtheorem{definition}{Definition}

%%%%%%%%%%%%%%%%%%%%%%%%
%%%%%%%%%%%%%%%%%%%%%%%%
%%%%%%Custom commands: mathbb
%%%%%%%%%%%%%%%%%%%%%%%%
%%%%%%%%%%%%%%%%%%%%%%%%

\newcommand{\A}{\mathbb A}
\newcommand{\C}{\mathbb{C}}
\newcommand{\D}{\mathbb{D}}
\newcommand{\E}{\mathbb{E}}
\newcommand{\F}{\mathbb{F}}
\newcommand{\N}{\mathbb{N}}
\renewcommand{\P}{\mathbb{P}}
\newcommand{\R}{\mathbb{R}}
\newcommand{\X}{\mathbb{X}}
\newcommand{\Z}{\mathbb{Z}}
\newcommand{\Q}{\mathbb{Q}}

%%%%%%%%%%%%%%%%%%%%%%%%
%%%%%%%%%%%%%%%%%%%%%%%%
%%%%%%Custom commands: greek
%%%%%%%%%%%%%%%%%%%%%%%%
%%%%%%%%%%%%%%%%%%%%%%%%

\renewcommand{\a}{\alpha}
\renewcommand{\b}{\beta}
\newcommand{\g}{\gamma}
\renewcommand{\d}{\delta}
\newcommand{\e}{\epsilon}
\renewcommand{\l}{\lambda}

\usepackage[utf8]{inputenc}

\title{Homework 9 \& 10}
\author{Sean Eva}
\date{October 2022}

\begin{document}

\maketitle

\begin{enumerate}
    \item [[\phantom{-}2]]
    
    \begin{enumerate}
        \item 
        
        Let $f(z) = e^z$, then we know that $f^{(n)}(z) = e^z$ for all $n\geq 0$, so then we know that $f^{(n)}(1) = e$ for all $n \geq 0$. Thus, $f(z) = e^z = \sum_{n = 0}^\infty \frac{f^{(n)}(1)}{n^1}(z-1)^n = e\sum_{n = 0}^\infty \frac{(z-1)^n}{n!}$ for $|z-1| < \infty.$
        
        \item
        
        Let us first replace $z$ with $z-1$ in the Maclaurin series for $e^z$, then we have that $e^{z-1} = \sum_{n = 1}^\infty \frac{(z-1)^n}{n!}$. Thus, we have that $e^z = e\sum_{n = 0}^\infty \frac{(z-1)^n}{n!}.$
        
    \end{enumerate}
    
    \item [[\phantom{-}10]]
    
    \begin{enumerate}
        \item 
        
        We will first recall that $\sinh(z) = \sum_{n = 0}^\infty \frac{z^{2n + 1}}{(2n + 1)!}$ when $|z| < \infty$. If $z \neq 0$ then we have that $\frac{\sinh(z)}{z^2} = \frac{1}{z^2}\sum_{n = 0}^\infty \frac{z^{2n + 1}}{(2n + 1)!} = \frac{1}{z^2}(z + \frac{z^3}{3!} + \frac{z^5}{5!} + ...) = \frac{1}{z} + \frac{z^2}{z*3!} + \frac{z^4}{z*5!} + ... = \frac{1}{z}(1 + \frac{z^1}{3!} + \frac{z^3}{5!} + ...) = \frac{1}{z}\sum_{n = 0}^\infty \frac{z^{2n + 1}}{(2n + 3)!}$ as desired.
        
        \item
        
        We will first recall that $\sin(z) = \sum_{n = 0}^\infty \frac{(-1)^nz^{2n + 1}}{(2n + 1)!}$ when $|z| < \infty$. If $z \neq 0$, then we have that $\frac{\sin(z^2)}{z^4} = \frac{1}{z^4}\sum_{n = 0}^\infty \frac{(-1)^nz^{4n + 2}}{(2n + 1)!} = \sum_{n = 0}^\infty \frac{(-1)^nz^{4n - 2}}{(2n + 1)!} = \frac{1}{z^2} - \frac{z^2}{3!} +...$ as desired.
        
    \end{enumerate}
    
    \item [[\phantom{-}4]]
    
    For $0 < |z| < 1$ we have that $\frac{1}{z^2(1 - z)} = \frac{1}{z^2}\sum_{n = 0}^\infty z^n = \sum_{n = 0}^\infty z^{n - 2} = \frac{1}{z^2} + \frac{1}{z} + \sum_{n = 2}^\infty z^{n - 2} = \sum_{n = 0}^\infty z^n + \frac{1}{z} + \frac{1}{z^2}$. Then we also have for $1 < |z| < \infty$, $\frac{1}{z^2(1 - z)} = -\frac{1}{z^3}\frac{1}{1 - \frac{1}{z}} = -\frac{1}{z^3}\sum_{n = 0}^\infty\frac{1}{z^n} = -\sum_{n = 0}^\infty \frac{1}{z^{n + 3}} = -\sum_{n = 3}^\infty\frac{1}{z^n}$ as desired. 
    
    \item [[\phantom{-}6]]
    
    First we have that $\frac{z}{(z - 1)(z - 3)} = \frac{1}{z-3}\frac{z}{z-1} = \frac{1}{z-3}\frac{z-1+1}{z-1} = \frac{1}{z - 3}(1 + \frac{1}{z - 1})$. Then for $|z - 1| < 2$, $\frac{1}{z - 3} = -\frac{1}{2 - (z - 1)} = -\frac{1}{2}{1}{1 - \frac{z-1}{2}} = -\frac{1}{2}\sum_{n = 0}^\infty \frac{(z - 1)^n}{2^n} = -\sum_{n = 0}^\infty\frac{(z - 1)^n}{2^{n + 1}}.$ Hence, $\frac{z}{(z - 1)(z - 3)} = -\sum_{n = 0}^\infty\frac{(z - 1)^n}{2^{n + 1}} - \sum_{n = 0}^\infty\frac{(z - 1)^{n - 1}}{2^{n + 1}} = -\frac{1}{2(z - 1)} - \sum_{n = 0}^\infty \frac{(z - 1)^n}{2^{n + 1}}-\sum_{n = 1}^\infty \frac{(z - 1)^{n - 1}}{2^{n + 1}} = -\frac{1}{2(z - 1)} - \sum_{n = 0}^\infty \frac{(z - 1)^n}{2^{n + 1}} - \sum_{n = 0}^\infty\frac{(z - 1)^n}{2^{n + 2}} = -\frac{1}{2(z - 1)} - 3\sum_{n = 0}^\infty \frac{(z - 1)^n}{2^{n + 2}}$ for $0 < |z - 1| < 2.$
    
    \item [[\phantom{-}2]]
    
    First we have that $\frac{1}{(1 - z)^2} = \sum_{n = 0}^\infty(n + 1)z^n$ for $|z| < 1$. If we replace $z$ with $\frac{1}{1 - z}$ we get, $\frac{1}{(1 - \frac{1}{1 - z})^2} = \sum_{n = 0}^\infty(n + 1)(\frac{1}{1 - z)}^n$. This is the same as $\frac{(1 - z)^2}{z^2} = \sum_{n = 0}^\infty \frac{(n = 1)}{(1 - z)^n}$ for $|\frac{1}{z - 1}| < 1$ or that is for $|z - 1| > 1.$ Thus, we have that $\frac{1}{z^2} = \sum_{n = 0}^\infty \frac{(n + 1)}{(1 - z)^{n + 2}} = \sum_{n = 2}^\infty\frac{(n - 1)}{(1 - z)^n} = \sum_{n = 2}^\infty\frac{(-1)^n(n - 1)}{(z - 1)^n}$ for $1 < |z - 1| < \infty$ as desired.
    
    \item [[\phantom{-}1]]
    
    We have $e^z = 1 + z + \frac{z^2}{2!} + ...$ and we also have that $\frac{1}{(z^2 + 1)} = 1 - z^2 + z^4 - z^6 + ...$. Thus we have that $\frac{e^z}{z(z^2 + 1)} = \frac{1}{z}(e^z)(\frac{1}{z^2 + 1}) = \frac{1}{z}(1 + z + \frac{z^2}{2!} + ...)(1 - z^2 + z^4 - z^6 + ...) = \frac{1}{z} + 1 + \frac{z}{2} - z ... = \frac{1}{z} + 1 - \frac{z}{2} - \frac{5}{6}z^2 + ...$ as desired.
    
    \item [[\phantom{-}3]]
    
    We have that $\int_C\frac{4z-5}{z(z-1)}dz = 2\pi iRes_{z = 0}(\frac{1}{z^2}f(\frac{1}{z})) = 2\pi i Res(\frac{-5z + 4}{z(1 - z)})$. Then we have that $\frac{-5z + 4}{z(1 - z)} = \frac{1}{z}\frac{-5z + 4}{(-z + 1)}$ for which the residue is $4$. Therefore, we have $\int_C\frac{4z-5}{z(z - 1)}dz = 2\pi i * 4 = 8\pi i.$
    
    \item [[\phantom{-}6]]
    
    Let the assumptions be as in the problem statement. That is that we have $f$ which is analytic over a closed contour $C$ except for a finite number of singular points within the contour. By Cauchy's Residue Theorem $\frac{1}{2\pi i}\int_Cf(z)dz = \sum_{z_1}^{z_n}R = Res_{z = z_1}f(z) + Res_{z = z_2}f(z) + ... Res_{z = z_n}f(z)$ and by the residue theorem at infinity, we have that since $f(z)$ is analytic on $|z| \geq R$. $-Res_{z = \infty}f(z) = \frac{1}{2\pi i}\int_{C}f(z)dz = -\frac{1}{2\pi i}\int_Cf(z)dz.$ Therefore we have that, $-Res_{z = \infty}f(z) = Res_{z = z_1}f(z) + Res_{z = z_2}f(z) + ... + Res_{z = z_n}f(z)$ which is the same as $Res_{z = z_1} + Res_{z = z_2} + ... + Res_{z = z_n} + Res_{z = \infty} = 0$ as desired.
    
\end{enumerate}

\end{document}
