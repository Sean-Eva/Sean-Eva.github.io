\documentclass[11pt, letter]{amsart}


\usepackage[margin=1in]{geometry}
\usepackage{amsthm}
\usepackage{amsmath}
\usepackage{amssymb}
\usepackage{enumerate}
\usepackage[inline]{enumitem}


\newtheorem*{theorem*}{Theorem}
\newtheorem*{lemma*}{Lemma}
\newtheorem*{claim*}{Claim}
\theoremstyle{definition}
\newtheorem{problem}{Problem}[]
\newtheorem{exercise}{Exercise}[]
\newtheorem*{definition*}{Definition}
\newcommand{\Prob}[1]{\ensuremath{%
    \mathbb P\left[#1\right]
}}
\newcommand{\ProbCond}[2]{\ensuremath{%
    \mathbb P\left[#1\:\middle|\:#2\right]
  }}
\newcommand{\Expect}[1]{\ensuremath{%
    \mathbb E\left[#1\right]
  }}
\newcommand{\p}[0]{\mathbb{P}}

\title[Math 4032: Homework \#5\qquad Due April 7 at 1:59pm]{Math 4032: Homework \#5\\
  Due April 7 at 1:59pm}


\begin{document}


\maketitle

The relevant background material for this assignment is covered in Chapters 18--20 of the Jukna book.  Techniques to be used in this assignment include the Lov\'asz Local Lemma and the Deletion / Alteration Method.

\begin{itemize}
\item You are strongly encouraged to typeset your homework solutions using \LaTeX.
\item Acknowledge collaborations as noted in the syllabus.
\item The following problems are optional exercises not to be turned in.  Problems to be turned in for a grade begin on the next page.  
\end{itemize}

Let $(\Omega, \mathbb P)$ be a finite probability space.  Recall that events $A_1, \dots, A_n \subseteq \Omega$ are \textit{mutually independent} if
\begin{equation*}
  \Prob{\bigcap_{i\in I}A_i} = \prod_{i\in I}\Prob{A_i} \quad\text{for every $I \subseteq [n]$}.
\end{equation*}
An event $A\subseteq \Omega$ is \textit{mutually independent} of events $A_1, \dots, A_n \subseteq \Omega$ if
\begin{equation*}
  \ProbCond{A}{\bigcap_{j\in J}B_j} = \Prob{A_i} \quad\text{for every $J \subseteq [n]$, where $B_j \in \{A_j, \overline{A_j}\}$}.
\end{equation*}
\begin{exercise}
  Prove that events $A_1, \dots, A_n \subseteq \Omega$ are mutually independent if and only if for every $i \in [n]$, the event $A_i$ is mutually independent of the events $A_1, \dots, A_{i-1}, A_{i+1}, \dots, A_n$.
\end{exercise}

  For finite probability spaces $(\Omega_1, \mathbb P_1), \dots, (\Omega_n, \mathbb P_n)$, the \textit{product space} of $(\Omega_1, \mathbb P_1), \dots,(\Omega_n, \mathbb P_n)$ has sample space $\Omega = \Omega_1 \times \cdots \times \Omega_n$ and probability distribution $\mathbb P$ where $\Prob{(\omega_1,\dots, \omega_n)} = \prod_{i=1}^n \mathbb P_i\left[\omega_i\right]$ for every $(\omega_1, \dots, \omega_n) \in \Omega$.  
\begin{exercise}
Let $(\Omega, \mathbb P)$ be the product space of $(\Omega_1, \mathbb P_1), \dots, (\Omega_m, \mathbb P_m)$, and let $A_1, \dots, A_n \subseteq \Omega$ be events, where each $A_i$ is \textit{determined} by the outcomes of $\bigcup_{i\in S_i} \Omega_i$ for some $S_i \subseteq [m]$ (i.e., if $(\omega_1, \dots, \omega_m) \in A_i$ and $(\omega'_1, \dots, \omega'_m) \in \Omega$ where $\omega_j = \omega'_j$ for $j \in S_i$, then $(\omega'_1, \dots, \omega'_m) \in A_i$).  Prove that if $J \subseteq [n]$ satisfies $S_i \cap S_j = \varnothing$ for all $j \in J$, then $A_i$ is mutually independent of $\{A_j : j \in J\}$.  \textit{Note: This fact is sometimes called the \textbf{Mutual Independence Principle}.  You can (and should) use it on the first three homework problems.}
\end{exercise}

\begin{exercise}
  Use the asymmetric version of the Lov\'asz Local Lemma to prove that the condition `$4pd \leq 1$' in the symmetric version of the Lov\'asz Local Lemma can be replaced with `$ep(d + 1)\leq1$'.  \textit{Hint: Choose each $x_i = 1 / (d + 1)$.}
\end{exercise}

\clearpage

Let $k,d \in \mathbb N$.  Recall that an \textit{arithmetic progression of length $k$ with common difference $d$} is a set of $k$ distinct numbers $a_1, \dots, a_k \in \mathbb N$ such that $a_{i+1} - a_i = d$ for all $i\in[k-1]$.  For example, $\{2, 5, 8, 11, 14\}$ is an arithmetic progression of length $5$ with common difference $3$.  An arithmetic progression of length $k$ is also called a \textit{$k$-AP}.

The \textit{van der Waerden number $W(2, k)$} is the smallest $n \in \mathbb N$ such that every $2$-coloring $\chi : [n] \rightarrow [2]$ of $[n]$ contains a monochromatic arithmetic progression of length $k$ (i.e., an arithmetic progression $S \subseteq [n]$ of length $k$ such that $S \subseteq \chi^{-1}(i)$ for some $i \in [2]$).  This problem concerns two different ways to prove lower bounds on $W(2, k)$.

\begin{problem}~
  \begin{enumerate}[label=(\alph*)]
  \item Prove that the number of $k$-APs in $[n]$ is less than $n^2  / k$, and prove that for every $k$-AP $S \subseteq [n]$ in $[n]$, fewer than $nk$ $k$-APs $S' \subseteq [n]$ in $[n]$ intersect $S$ (i.e.\ satisfy $S\cap S' \neq \varnothing$).

    \begin{claim*}
        The number of $k$-APs in $[n]$ is less than $n^2 / k$
    \end{claim*}
    \begin{proof}
        Let assumptions be as in the problem statement. Let us denote $j$ as the smallest element in some arithmetic progression in $[n]$. We know then that the value of $j$ is within $1 \leq j \leq n - k + 1$. For such arithmetic progression, let us say that the common difference is $d$, then we have that $j + (k - 1)d \leq n$ since the $k$ elements need to fit in $[n]$. This then implies that $1 \leq d \leq \frac{n - j}{k - 1}$. Since we know that $d$ needs to be an integer we know then that we have $\lfloor\frac{n-j}{k-1}\rfloor$ values for $d$. And similarly, we find that this is the same number of arithmetic progressions that would begin with $j$. Thus we find that the total number of arithmetic progressions is $\sum_{j = 1}^{n - k + 1} \lfloor\frac{n - j}{k - 1}\rfloor = \sum_{j = k - 1}^{n - 1}\lfloor\frac{j}{k - 1}\rfloor = \sum_{j = 1}^{n - 1}\lfloor\frac{j}{k - 1}\rfloor \leq \sum_{j = 1}^{n - 1} \frac{j}{k - 1} = \frac{1}{k - 1} + \frac{2}{k - 1} + ... + \frac{n - 1}{k - 1} = \frac{1 + 2 + ... + (n - 1)}{k - 1} = \frac{n (n - 1)}{2(k - 1)} < \frac{n^2}{k}$. Thus, we have that the number of $k$-APs in $[n]$ is less than $n^2 / k$ as desired.
    \end{proof}
    \begin{claim*}
        For every $k$-AP $S \subseteq [n]$ in $[n]$, fewer than $nk$ $k$-APs $S' \subseteq [n]$ in $[n]$ intersect $S.$
    \end{claim*}
    \begin{proof}
        This claim is the same as asking how many have a fixed number which we will denote as $x \in [n]$. Let us say that this arbitrary $x$ is the $i$th element of some $k$-AP with some common difference $d$. This then implies that $1 \leq x - (i - 1)d$ and $x + (k - i) d \leq n$. Let us assume then that $k$ is even. We will sum over all $i$ while assuming $x + (k - i)d \leq n$ for all $i \leq k/2$ and then assuming $1 \leq x - (i - 1)d$ for $i > k/2$ to get the following bound. $\sum_{i = 1}^{k/2}\frac{n - x}{k - i} + \sum_{i = k/2 + 1}^k\frac{x - 1}{i - 1} = (n - x)\sum_{i = 1}^{k/2}\frac{1}{k - i} + (x - 1)\sum_{i = k/2 + 1}^k\frac{1}{i - 1} = (n - x + x - 1)\sum_{i = k/2 + 1}^k\frac{1}{i - 1} \leq n - 1$. The argument for if $k$ is odd is similar. Then we can apply this to all $k$ elements of the $k$-AP to find that $k(n - 1) < kn$ as desired.
    \end{proof}
  
  \item Prove that $W(2, k) > \sqrt{k 2^{k - 1}}$ using the Union Bound.

    \begin{proof}
        Let assumptions be as in the problem statement. If we randomly color all elements of $[n]$ either $0$ or $1$ it is simple to see the probability of forming a $k$-AP is $2^{-(k-1)}$ since the first element doesn't matter just that the other $k - 1$ elements match it. Then from above, we know that the number of $k$-APs is bounded by $n^2/k$ and if we take the union bound over each of these $k$-APs, we get $\frac{n^2}{k2^{k-1}}$. We desire this to be $< 1$ which would then imply that $n < \sqrt{k2^{k-1}}$. This then implies that $W(2, k) > \sqrt{k2^{k-1}}$ as desired for sufficiently large n. 
    \end{proof}
  
  \item Prove that $W(2, k) > 2^k / (2ek)$ using the Lov\'asz Local Lemma (also, see Exercise 3).

    \begin{proof}
        Let assumptions be as in the problem statement. Consider a random $2$-coloring of $[n] = [\frac{2^k}{2ek}$ uniformly and independently. Let us define a graph $G = (V, E)$ with $V = \{A_1, ..., A_m\}$ where $m$ is the number of $k$-APs in $[\frac{2^k}{2ek}]$. Let $A_i$ be the event that the $i$th $k$-AP is monochromatic. Let there be an edge, $e\in E$, between $A_i$ and $A_j$ where $i\neq j$ when their $k$-APs intersect, i.e. these event are dependent on each other when they intersect and thus result in an edge on a dependency graph. Thus, $G$ is a dependency graph on the set of $A_i$s where we know that the set of $A_i$s and its non-neighbors in the dependency graph are mutually independent. It is simple to see that $\p(A_i) = \frac{1}{2^{k-1}}$ which is just the probability that it is monochromatic for all $i$. Then we find that the maximum degree is $nk-1$ which is the number of $k$-APs that intersect $A_i$ minus $A_i$ itself as from part $a$. Then by the LLL and from excercise 3 we have that $ep(d + 1) = e * \frac{1}{2^{k - 1}} (\frac{k2^k}{2ek} - 1 + 1) = 1 = 1 \checkmark$. Thus we know for $n = \frac{2^k}{2ek}$ then there exists a $2$-coloring thus we know that $W(2, k) > \frac{2^k}{2ek}$ as desired.
    \end{proof}
  
  \end{enumerate}
\end{problem}

\clearpage
\begin{problem}
  Let $G = (V_1 \cup \cdots \cup V_m, E)$ be a graph, where $V_1, \dots, V_m$ are pairwise disjoint.  Prove that if every vertex of $G$ has degree at most $d$ and $|V_i| \geq 2ed$ for every $i \in [m]$, then there exists an independent set $\{u_1, \dots, u_m\}$ in $G$ where $u_i \in V_i$ for every $i \in [m]$.  \textit{Hint: Choose $u_i \in V_i$ uniformly at random, and use the Lov\'asz Local Lemma (also, see Exercise 3).}
\end{problem}

\begin{proof}
    Let assumptions be as in the problem statement. Let us choose $u_i \in V_i$ uniformly at random for $i\in [m]$, and then we will define $I = \{u_1, ..., u_m\}$ be a prospective independent set. Let $A_e$ be the event that for some edge $e\in E$ that both of its end vertices are in $I$. Since the $u_i$ are selected at random and we know from the problem statement that $|V_i| \geq 2ed$, then we know that $\p(A_e) \leq \frac{1}{(2ed)^2}$. We will define the dependency graph on the events $A_e$ such that two events are dependent on each other if they have a dependency between the same two $V_i$s for which then we know that the maximum degree is $|V_1|d+|V_2|d - 1 = 4ed^2 - 1$. If we apply the LLL and the condition from exercise 3 then we find that $e * \frac{1}{(2ed)^2} * (4ed^2 - 1 + 1) = 1 \leq 1 \checkmark$. This by the LLL we know that there does exist an independent set $\{u_1, ..., u_m\}$ in $G$ where $u_i\in V_1$ for every $i \in [m]$ as desired.
\end{proof}

\clearpage

\begin{problem}
  Let $s,t,n \in \mathbb N$.  Recall Problem 2 from HW \#4.  The following problem provides an improvement using the Local Lemma and the Deletion Method.
  \begin{enumerate}[label=(\alph*)]
  \item\label{ramsey-symmetricLLL} Prove that if $e\binom{t}{2}\binom{n - 2}{t - 2}2^{1 - \binom{t}{2}} < 1$, then $R(t, t) > n$.  \textit{Hint: Use the Lov\'asz Local Lemma (also, see Exercise 3)}.

    \begin{proof}
        Let assumptions be as in the problem statement. Let us uniformly at random consider a $2$-coloring of the edges of a $K_n$. For each subset $S\subset [n]$ of vertices of size $t$, we will let $A_S$ be considered the bad events for the LLL where $S$ is monochromatic. Then it is simple to see that $\p (A_S) = 2^{1-\binom{k}{2}}$. For the dependency graph on such $A$ are dependent if for two events $A_{S_1}, A_{S_2}$ that $|S_1\cap S_2| \geq 2$; i.e. when $S_1$ and $S_2$ share one edge with each other. Thus, for a fixed $S$, there are at most $\binom{t}{2}\binom{n-2}{t-2}S_2$ such that $|S_1\cap S_2| \geq 2$. Therefore, the maximum degree of the dependency graph is such that $d \leq \binom{t}{2}\binom{n-2}{t-2}$ and by LLL and the problem statement we know that this is $ep(d+1) < 1$ and we know that $R(t, t) > n$ as desired.
    \end{proof}

  \item\label{ramsey-asymmetricLLL} Prove that, if there exists $p,x,y\in (0, 1)$ satisfying
    \begin{equation*}
      p^{\binom{s}{2}} \leq x(1 - x)^{\binom{s}{2}\binom{n-2}{s - 2}}(1 - y)^{\binom{n}{t}}\quad\text{and}\quad(1 - p)^{\binom{t}{2}} \leq y (1 - x)^{\binom{t}{2} \binom{n-2}{s-2}}(1 - y)^{\binom{n}{t}},
    \end{equation*}
    then $R(s, t) > n$.  \textit{Hint: Consider the random graph $G_{n,p}$ and use the asymmetric Lov\'asz Local Lemma with $x$ and $y$ playing the roles of $x_i$}

    \begin{proof}
        I'm a little stumped on this problem, I'm gonna to write notes about how I expect to solve it. I expect I would follow a similar construction as in (a) but when I apply asymmetric LLL we will take the probability of the intersection of the disjoint events to find the product in the problem statement.
    \end{proof}
    
  \item\label{ramsey-deletion} Prove that, for every $p \in [0,1]$,
    \begin{equation*}
      R(s,t) > n - \binom{n}{s}p^{\binom{s}{2}} - \binom{n}{t}(1 - p)^{\binom{t}{2}}.
    \end{equation*}
    \textit{Hint: Consider the graph obtained from the random graph $G_{n,p}$ by deleting vertices from monochromatic cliques.}

    \begin{proof}
        Consider a random graph $G_{n, p}$ with probability $p$ that there exists an edge between two vertices in $G$. Let us define a random variable $X$ to be the number of $s$-cliques plus the number of $t$-independent sets in $G$. Then we find that $\mathbb{E}[X] = \mathbb{E}[$number of $s$-cliques$] + \mathbb{E} [$number of $t$-independent sets$] = \sum_{s\text{-subsets}\in G}\p(\text{the s-subset forms a clique}) + \sum_{\text{t-subsets}\in G}\p(\text{the t-subset forms an independent set}) = \binom{n}{2}p^{\binom{s}{2}} + \binom{n}{t}(1-p)^{\binom{t}{2}}.$ By the pigeonhole for expectation we have that there exists a graph with $|V| = n$ with at most $\binom{n}{s}p^{\binom{s}{2}} + \binom{n}{t}(1-p)^{\binom{t}{2}}$ s-cliques and t-independent sets. The resulting graph has at least $n - \binom{n}{s}p^{\binom{s}{2}} - \binom{n}{t}(1-p)^{\binom{t}{2}}$ vertices but no s-cliques and no t-independent sets. So this can serve as a lower bound for $R(s, t)$ as desired in the problem statement.
    \end{proof}
    
  \end{enumerate}
\end{problem}
\clearpage
\begin{problem}The following problem uses the previous problem to obtain lower bounds on Ramsey numbers.  Note the difference in bounds obtained via the Local Lemma (Problem 3(a--b)), the Deletion Method (Problem 3(c)), and the Union Bound (Problem 2(a) of HW \# 4).
  \begin{enumerate}[label=(\alph*)]
  \item Prove the following bounds on $R(k,k)$ using Problem 2(a) of HW\#4, Problem 3\ref{ramsey-deletion}, and Problem 3\ref{ramsey-symmetricLLL}, respectively:
    \begin{itemize}
    \item $R(k,k) > (1 - o(1))k2^{(k-1)/2} / e$,


    
    \item $R(k,k) > (1 - o(1))k2^{k/2} / e$,

    
        
    \item $R(k,k) > (1 - o(1))k2^{(k+1)/2} / e$.

    
    
    \end{itemize}
  \item Prove that $R(3, t) = \Omega(t^2 / \log^2 t)$ using Problem 3\ref{ramsey-asymmetricLLL}.  \textit{Hint: Consider $p = c_1 / \sqrt n$, $x = c_2 / n^{3/2}$, and $y = c_3 / \binom{n}{t}$, for appropriately chosen constants $c_1, c_2, c_3 \in \mathbb  R$}.

    
  
  \item Prove that $R(4, t) = \Omega(t^2 / \log^2 t)$ using Problem 3\ref{ramsey-deletion}.  \textit{Note: This result also follows from 4(b).  A similar approach would give an even stronger bound of $R(4,t) = \Omega((t / \log t)^{5/2})$.}
  \end{enumerate}
\end{problem}

\clearpage
For every $s, n \in \mathbb N$, let $k_s(n)$ be the minimum $m \in \mathbb N$ such that every bipartite graph $G = (A\cup B, E)$ with $|A| = |B| = n$ and $|E| > m$ contains a subgraph isomorphic to $K_{s,s}$.  Recall that the K\H{o}v\'ari--S\'os--Tur\'an Theorem implies that $k_s(n) \leq (s - 1)^{1/s}n^{2 - 1/s} + (s - 1)n$.  
\begin{problem}
  This problem uses the Deletion Method to provide a lower bound on $k_s(n)$.
  \begin{enumerate}[label=(\alph*)]
  \item Prove that for every $p \in [0,1]$ and $s,n \in \mathbb N$,
    \begin{equation*}
      k_s(n) \geq n^2p - \binom{n}{s}^2 p^{s^2}.
    \end{equation*}

    \begin{proof}
        Let assumptions be as in the problem statement. Let us consider a random graph $G = (V, E)$ where $V = A\cup B$ and $|A| = |B| = n$ such that no vertices in $A$ are adjacent to other vertices in $A$ and similarly for $B$ to form a bipartite graph as in the problem statement. Let us define a random variable $X$ to be the number of edges in the graph $G$ minus the number of $K_{s,s}$ subgraphs of $G$. The expected value of this random variable is the expected number of edges minus the expected number of $K_{s,s}$. By the linearity of expectation, this reduces to $n^2p-\binom{n}{s}^2p^{s^2}$. By the pigeonhole for expectation, there exists a graph with its random variable $X$ less than $n^2p-\binom{n}{s}^2p^{s^2}$. Let us now consider deleting an edge from each $K_{s,s}$ in such a graph to remove all of them. The number of edges remaining would be $n^2p - \binom{n}{s}^2p^{s^2}$. Thus, we know there exists a graph $G' = (V, E)$ where $|E| = n^2p-\binom{n}{s}^2p^{s^2}$ so we know that $k_s(n) > n^2p-\binom{n}{s}^2p^{s^2}$ as desired.
    \end{proof}
    
  \item   Let $s \in \mathbb N$.  Prove that $k_s(n) = \Omega(n^{2 - 2/s})$.

    \begin{proof}
        Let assumptions be as in the problem statement. Let us consider a random graph $G = (V, E)$ the same as in part a. Using the similar logic, we will note that $\binom{n}{s} < n^s$ and choose $p = n^{-2/s}$. This allows us to see that $n^2p - \binom{n}{s}^2 p^{s^2} \geq c * n^{2-2/s}$ for some $c > 0$. Thus, we have that $k_s(n) \geq cn^{2-2/s}$ for $c > 0$, i.e. $k_s(n) = \Omega(n^{2-2/s})$ as desired.
    \end{proof}
  
  \end{enumerate}

  
\end{problem}

\end{document}

%%% Local Variables:
%%% mode: latex
%%% TeX-master: t
%%% End:
