\documentclass[letterpaper,12pt,addpoints]{exam}
\usepackage[utf8]{inputenc}
\usepackage[english]{babel}
\usepackage{amsthm}
\usepackage{amssymb}
\usepackage{mathcomp}
\usepackage{amsmath}
\usepackage{natbib}
\usepackage{array}
\usepackage{wrapfig}
\usepackage{multirow}
\usepackage{tabularx}
\usepackage{epic}

\qformat{Question \thequestion\dotfill \emph{\totalpoints\ point}}
\newtheorem{ishaan}{Theorem}[section]
\newtheorem{lemma}{Lemma}[section]
\renewcommand\qedsymbol{$\blacksquare$}
\renewcommand{\labelenumii}{\arabic{enumii}.}

\title{Midterm 2}
\author{Sean Eva}
\date{April 2021}

\begin{document}

\header{MATH 3235}{Midterm 2}{Due April 6, 2021 before class}
\firstpagefooter{}{}{}
\pagestyle{headandfoot}
\vspace*{1cm}
\begin{center}
\fbox{\parbox{6in}{\centering {\bf This is a take home midterm. You can use 
your notes, my online notes on canvas and the textbooks book. 
You are supposed to work on your own text without external help. I'll be 
available to answer question in person or via email. Please, write clearly and 
legibly and take a readable scan before uploading.}}}
\end{center}


\vspace*{\fill}

\vspace{0.5cm}
\noindent
\makebox[\textwidth]{Name (print): Sean Eva}
\vspace{0.5cm}

\vspace*{\fill}
 
\begin{center}
\gradetable[h][questions]
\end{center}
% \vspace{0.5cm}
% \begin{center}
% \bonusgradetable[h][questions]
% \end{center}
\vspace{0.5cm}

\vspace*{\fill}
\newpage


\printanswers

\setcounter{page}{1}
\footer{}{Page \thepage\ of \numpages}{}

\begin{questions}
\question
Let $X$ be a normal random variable with $\mu=\mathbb{E}(X)=2$ and 
$\sigma^2=\mathrm{var}(X)=4$.

\begin{parts}
\part[10]
Compute $\mathbb{P}(X<0)$. Express the result in term of the cumulative 
distribution function $\Phi$ of a Normal Standard r.v.. $\Phi$ is usually 
called the {\it probability integral}.

\begin{solution}
Given that $\mu = \mathbb{E}(x)=2$ and $\sigma^2=var(x)=4, \sigma=2.$ Therefore,
\begin{align*}
    \mathbb{P}(X<0)&=\mathbb{P}(Z<\frac{0-2}{2})\\
    &=\mathbb{P}(Z<-1)\\
    &=\Phi(-1)
\end{align*}
\end{solution}
\vspace*{\fill}

\part[10]
Find $\delta$ such that
\[
 \mathbb{P}(2-\delta<X<2+\delta)=0.95\,.
\]
Express the result in term of the $\alpha$ critical value $z_\alpha$ defined as 
$\Phi(-z_\alpha)=\alpha$.
\begin{solution}
\begin{align*}
    \mathbb{P}(2-\delta<X<2+\delta)&=0.95\\
    \mathbb{P}(\frac{2-\delta-2}{2}<Z<\frac{2+\delta-2}{2}&=0.95\\
    \mathbb{P}(\frac{-\delta}{2}<Z<\frac{\delta}{2})&=0.95\\
    \Phi(\frac{\delta}{2})-\Phi(\frac{-\delta}{2})&=0.95\\
    \frac{\delta}{2}&=1.96\\
    \delta&=3.9 2
\end{align*}
\end{solution}
\end{parts}
\vspace*{\fill}\eject

%%%%%%%%%%%%%%%%%%%%%%%%%%%%%%%%%%%%%%%%%%%%%%%%%%%%%%%%%%%%
\question

Let $X$ and $Y$ be two r.v. such that the marginal p.d.f. of $X$ is
\[
 f_X(x)=
 \begin{cases}
  4x e^{-2x} & x>0\\
  0          & x\leq 0
 \end{cases}
\]
and the conditional p.d.f. of $Y$ given $X$ is
\[
 f_{Y|X}(y|x)=
 \begin{cases} 
  \frac1x & 0<y<x\\
  0 & \mathrm{otherwise}\,.
 \end{cases}
\]
This means that, given $X=x$, $Y$ in uniform in $[0,x]$.
\begin{parts}

\part[10]
Write the joint p.d.f. $f(x,y)$ of $X$ and $Y$.
\begin{solution}
\begin{align*}
    f_{X,Y}(x, y)&=f_{Y|X}(y|x)f_X(x)\\
    &=\frac{1}{x}4xe^{-2x}\\
    &=\begin{cases}
     4e^{-2x} & 0<y<x<\infty \crcr
     0 & \text{otherwise}
    \end{cases}
\end{align*}
\end{solution}
\vspace*{\fill}

\part[10]
Compute the marginal p.d.f. of $f_Y(y)$ of $Y$ and the conditional p.d.f. 
$f_{X|Y}(x|y)$ of $X$ given $Y$.

\begin{solution}
\begin{align*}
    f_Y(y)&=\int_y^{\infty}f_{X,Y}(x,y)dx\\
    &=\int_y^{\infty}4e^{-2x}\\
    &=4[\frac{e^{-2x}}{-2}]^{\infty}_y\\
    &=\begin{cases}
     2e^{-2y}&y>0\crcr
     0 & \text{otherwise}
    \end{cases}\\
    f_{X|Y}(x|y)&=\frac{f_{X,Y}(x,y)}{f_Y(y)}\\
    &=\frac{4e^{-2x}}{2e^{-2y}}\\
    &=\begin{cases}
     2e^{-2(x-y)}&x>y\crcr
     0&\text{otherwise}
    \end{cases}
\end{align*}
\end{solution}
\vspace*{\fill}
\eject


\part[15]
Compute $\mathbb{P}(Y>X/2)$.({\bf Hint}: consider first 
$\mathbb{P}(Y>X/2|X=x)$.)

\begin{solution}
\begin{align*}
    \mathbb{P}(Y>\frac{x}{2}|X=x)&=\int_{\frac{x}{2}}^xf_{Y|X}(y|x)dy\\
    &=\int_{\frac{x}{2}}^x\frac{1}{x}dy\\
    &=\frac{1}{x}(x-\frac{x}{2})\\
    &=\frac{1}{2}.
\end{align*} Therefore,
\begin{align*}
\mathbb{P}(Y>\frac{X}{2})&=\int_0^{\infty}\mathbb{P}(Y>\frac{x}{2}|X=x)f_X(x)dx\\
&=\int_0^{\infty}\frac{1}{2}4xe^{-2x}dx\\
&=2[x\int e^{-2x}dx-\int\frac{d}{dx}x\int e^{-2x}dxdx]^{\infty}_0\\
&=2[x\frac{e^{-2x}}{-2}-\frac{e^{-2x}}{4}]^{\infty}_0\\
&=\frac{2}{4}\\
&=\frac{1}{2}
\end{align*}
\end{solution}

\end{parts}

\vspace*{\fill}\eject


%%%%%%%%%%%%%%%%%%%%%%%%%%%%%%%%%%%%%%%%%%%%%%%%%%%%%%%%%%%%%%


\question[15]
Let $X$ and $Y$ be two independent Normal Standard r.v., that is 
the joint p.d.f. of $X$ and $Y$ is
\[
 f_{X,Y}(x,y)=\frac 1{2\pi} e^{-\frac{x^2+y^2}2}\, .
\]

Call
\[
 \begin{aligned}  
  U&=X+Y\\
  V&=X-Y\,.
 \end{aligned}
\]
Compute the joint p.d.f. of $U$ and $V$. Are they independent?


\begin{solution}
We could rewrite our terms of $U$ and $V$ as $X=\frac{U+V}{2}, Y=\frac{V-U}{2}$. Therefore,
\begin{align*}
    f_{X,Y}(x,y)&=\frac 1{2\pi} e^{-\frac{x^2+y^2}2}\\
    f_{U,V}(u,v)&=\frac{1}{2\pi}e^{-\frac{(\frac{u+v}{2})^2+(\frac{v-u}{2})^2}{2}}\\
    &=\frac{1}{2\pi}e^{-\frac{(\frac{u^2+2uv+v^2}{4})+(\frac{v^2-2uv+u^2}{4})}{2}}\\
    &=\frac{1}{2\pi}e^{-\frac{(\frac{u^2+v^2}{2})}{2}}\\
    &=\frac{1}{2\pi}e^{-\frac{u^2+v^2}{4}}.
\end{align*} $U$ and $V$ are independent if this joint p.d.f. can be represented as a product of a p.d.f. in terms of $u$ and a p.d.f. in terms of $v$. This is possible as we could write $f_U(u)=\frac{1}{2\pi}e^{\frac{u^2}{-4}}$ and $f_V(v)=e^{\frac{v^2}{-4}}$. Since $f_U(u)*f_V(v)=f_{U,V}(u,v)$, we can say that $U$ and $V$ are independent.
\end{solution}

\vspace*{\fill}\eject




%%%%%%%%%%%%%%%%%%%%%%%%%%%%%%%%%%%%%%%%%%%%%%%%%%%%%%%%%%%%%%%%%%%%%%%%%%%%%%


\question[15]
If $X$ is a continuous r.v., the upper quintile $q(0.8)$ of the p.d.f. of $X$ 
is defined by
\[
 \mathbb{P}(X\leq q(0.8))=0.8.
\]
A Pareto r.v. $X$ with shape $\alpha$ is defined by the p.d.f.
\[
 f(x)=\begin{cases}
 \alpha x^{-(\alpha+1)}& x\geq1\\
 0 & x<1
 \end{cases}
\]
where $\alpha>1$.

Compute $q(0.8)$ when $X$ is a Pareto r.v. with shape 
$\alpha$.


\begin{solution}
pdf is given by $f(x)=\begin{cases}
 \alpha x^{-(\alpha+1)}&x\geq1\crcr
 0&x<1
\end{cases}$.
 Now,
\begin{align*}
    \int_{-\infty}^{\infty}f(x)dx&=1\\
    &\Rightarrow \int_{-\infty}^10*dx+\int_1^{\infty}\alpha x^{-(\alpha+1}dx\\
    &=1\\
    &\Rightarrow [\alpha \frac{x^{-\alpha-1+1}}{-\alpha-1+1}]^{\infty}_1\\
    &=-[\frac{1}{x^{\alpha}}]^{\infty}_1\\
    &= -[0-1]\\
    &=1.
\end{align*} Now, $\mathbb{P}(X\leq q(0.8))$. Let $q(0.8)=x$. So,
\begin{align*}
    \mathbb{P}(X\leq x)&=0.8\\
    &\Rightarrow \int_1^xf(x) = 0.8\\
    &\Rightarrow -[\frac{1}{x^\alpha}]^x_1=0.8\\
    &\Rightarrow \frac{-1}{x^\alpha}+1=0.8\\
    &\Rightarrow \frac{1}{x^\alpha}=0.2\\
    &\Rightarrow x^\alpha = 5\\
    &\Rightarrow x=5^{\frac{1}{\alpha}}\\
    &\Rightarrow q(0.8)=5^{\frac{1}{\alpha}}  \hspace{1cm} (\alpha>1)
\end{align*}
\end{solution}


\vspace*{\fill}\eject

\question[15]
Let $X_1$ and $X_2$ be two independent r.v. uniformly distributed in $[0,1]$. 
Find the p.d.f. of $Y=X_1+X_2$.

\begin{solution}
$f_Y(y)=\int_{-\infty}^{\infty}f_{X_1}(x)f_{X_2}(y-x)dx.$ We could denote both $f_{X_1}$ and $f_{X_2}$ as $f_X$ since they both have the exact same distribution. Therefore, we could rewrite $f_Y(y)$ as $f_Y(y)=\int_{-\infty}^{\infty}f_X(x)f_X(y-x)dx.$ The integrand of this function can have values $1$ when $0<x<1$ and $0<y-x<1$ and $0$ otherwise. The limits of integration will depend on the value of $y.$
\begin{enumerate}
    \item 
    
    When $0<y<1$, $x=0$ to $x=y$. So $f_Y(y)=\int_0^y1dx=y.$
    
    \item
    
    When $1<y<2,x=y-1$ to $x=1,$  so $f_Y(y)=\int_{y-1}^1 1dx=2-y.$
    
    \item
    
    When $y<0$ or $y>2$, the integrand is zero, so $f_Y(y)=0$
    
\end{enumerate}
Therefore, the p.d.f. of $Y$ is $f_Y(y)=\begin{cases}
 y&0<y<1\crcr
 2-y&1\leq y<2\crcr
 0&\text{otherwise}
\end{cases}$
\end{solution}

\end{questions}


\end{document}
