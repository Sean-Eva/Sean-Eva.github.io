\documentclass{article}
\usepackage[utf8]{inputenc}
\usepackage[english]{babel}
\usepackage{amsthm}
\usepackage{amssymb}
\usepackage{mathcomp}
\usepackage{amsmath}
\usepackage{natbib}
\usepackage{array}
\usepackage{wrapfig}
\usepackage{multirow}
\usepackage{tabularx}

\newtheorem{ishaan}{Theorem}[section]
\newtheorem{lemma}{Lemma}[section]
\renewcommand\qedsymbol{$\blacksquare$}

\title{Homework 2}
\author{Sean Eva}
\date{September 2021}

\begin{document}

\maketitle

\begin{enumerate}
    \item 
    
    \begin{enumerate}
        \item 
        
        If the hydrogen begins are the $n=2$ state , it begins with $E_2=13.6(1-\frac{1}{2^2})=10.2$eV. In order for the hydrogen to become ionized it needs to have $13.6$eV of energy. This means that it needs $13.6-10.2=3.4$eV of energy to become ionized.
        
        \item
        
        Formula: $E=hv=\frac{hc}{\lambda}$ where $h$ is Planck's Constant.\\
        \begin{align*}
            3.4eV &= 5.44*10^{-19}\\
            5.44*10^{-19} &= 6.625*10^{-34}v\\
            8.211*10^{14}Hz &= v\\
            c &= v\lambda\\
            3*10^8 &= (8.211*10^{14})\lambda\\
            365\text{nm} &= \lambda
        \end{align*}
        
    \end{enumerate}
    
    \item
    
    \begin{enumerate}
        \item 
        
        \begin{align*}
            \Delta E &= \frac{hc}{\lambda}\\
            \lambda &= \frac{hc}{\Delta E}\\
            \frac{1}{\lambda} &= RZ^2(\frac{1}{m^2}-\frac{1}{n^2})\\
            \Delta E &= RhcZ^2(\frac{1}{m^2}-\frac{1}{n^2})\\
            \Delta E &= (1.097*10^7)(3.625*10^{-34})(3*10^8)(\frac{1}{1^2}-\frac{1}{4^2})\\
            \Delta E &= 1.118*10^{-18}\text{J} = 6.988\text{eV}
        \end{align*}
        
        \item
        
        Since the energy for emission is the same as the energy for absorption the photon would need the same amount of energy of $6.998$eV.
        
    \end{enumerate}
    
    \item
    
    Energy to ionize a hydrogen atom: $13.6$eV. The energy at $n=3$ is $E_n = 13.6\text{eV}(1-\frac{1}{n^2}) = 13.6(1-\frac{1}{9}) = 12.1\text{eV}$. Therefore, the photon would need $13.6-12.1=1.5\text{eV}$.\\
    \begin{align*}
        1.5*(1.6*10^{-19}) &= \frac{(3.625*10^{-34})(3*10^8)}{\lambda}\\
        453\text{nm} &= \lambda
    \end{align*}
    
    \item
    
    Rydberg Formula: $\frac{1}{\lambda}=RZ^2(\frac{1}{m^2}-\frac{1}{n^2})$
    \begin{align*}
        \frac{1}{468.6*10^{-9}} &= (1.097*10^{7})(2)^2(\frac{1}{m^2}-\frac{1}{n^2})\\
         &= (\frac{1}{m^2}-\frac{1}{n^2}).
    \end{align*} If $m=3$ and $n=4$. Then this equality is true. Therefore, this electronic transition is the Paschen $\alpha$ line.
    
    \item
    
    The minimum amount of energy to excite the atom would be to move it from the ground state to $n=2$. The amount of energy to cause this change would be $10.2$eV.
    
    \item
    
    \begin{enumerate}
        \item 
        
        $5.1\text{eV} = 8.16*10^{-19}$
        \begin{align*}
            8.16*10^{-19} &- \frac{1}{2}mv^2\\
            &= \frac{1}{2}(9.11*10^{-31}*v^2\\
            1.33*10^6 \frac{m}{s} &= v.
        \end{align*}
        
        \item
        
        \begin{align*}
            8.16*10^{-19} &- \frac{1}{2}mv^2\\
            &= \frac{1}{2}(1.67*10^{-27}*v^2\\
            9.77*10^8 \frac{m}{s} &= v.
        \end{align*}
        
    \end{enumerate}
    
\end{enumerate}

\end{document}
