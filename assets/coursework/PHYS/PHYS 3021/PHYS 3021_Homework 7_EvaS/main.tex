\documentclass{article}
\usepackage[utf8]{inputenc}
\usepackage[english]{babel}
\usepackage{amsthm}
\usepackage{amssymb}
\usepackage{mathcomp}
\usepackage{amsmath}
\usepackage{natbib}
\usepackage{array}
\usepackage{wrapfig}
\usepackage{multirow}
\usepackage{tabularx}
\usepackage{multirow}
\usepackage{graphicx}

\newtheorem{ishaan}{Theorem}[section]
\newtheorem{lemma}{Lemma}[section]
\renewcommand\qedsymbol{$\blacksquare$}

\title{PHYS 3021}
\author{Sean Eva}
\date{September 2021}

\begin{document}

\maketitle

\begin{enumerate}
    \item 
    
    Stefan-Boltzmann Law: $E=4\pi R^2 \sigma T^4$
    \begin{align*}
        \frac{E_{\text{giant}}}{E_{\text{hotter}}} &= (4)^2(\frac{7500}{12000})^4\\
        \frac{E_{\text{giant}}}{E_{\text{hotter}}} &= 2.44.
    \end{align*} The giant star is $2.44$ times as luminous as the smaller, but hotter star is.
    
    \item
    
    Kepler's Third Law: $(M_1+M_2)P^2 = a^3$
    \begin{align*}
        (M_1+M_2) &= \frac{a^3}{P^2}\\
        (M_1+M_2) &= \frac{(\tan(\frac{2}{3600})*(6.25*206265))^3}{(17)^2}\\
        (M_1+M_2) &= 6.76\text{ solar masses}
    \end{align*}
    
    \item
    
    \begin{enumerate}
        \item 
        
        Spectroscopic Binaries: $\frac{M_1}{M_2}=\frac{v_2}{v_1}$
        \begin{align*}
            \frac{M_1}{M_2} &= \frac{500}{100}\\
            \frac{M_1}{M_2} &= 5.
        \end{align*} One star is 5 times more massive than the other.
        
        \item
        
        Kepler's Third Law: $(M_1+M_2)P^2 = a^3$ where $(M_1+M_2)=6M_2$ since $M_1=5M_2$
        \begin{align*}
            (6M_2) &= \frac{a^3}{P^2}\\
            (6M_2) &= \frac{(0.21)^3}{(0.04)^2}\\
            (6M_2) &= 5.71 \text{ solar masses}\\
            M_1 &= 4.82 \text{ solar masses}\\
            M_2 &= 0.96 \text{ solar masses}
        \end{align*}
        
    \end{enumerate}
    
    \item
    
    \begin{enumerate}
        \item 
        
        \begin{align*}
            \log(\frac{L_1}{L_2}) &= 0.4(m_2-M_1)\\
            \frac{L_1}{L_2} &= 10^{0.4(m_2-m_1)}\\
            &= 10^{0.4(11.5-1.8)}\\
            &= 7.59*10^3.
        \end{align*}
        
        \item
        
        Stefan-Boltzmann Law: $L=e\pi R^2\sigma T^4$
        \begin{align*}
            \frac{T_1}{T_2} &= (\frac{L_1}{L_2}(\frac{R_1}{R_1})^2)^{1/4}\\
            &= (7.59*10^3(\frac{0.01}{1.4})^2)^{1/4}\\
            &= 0.789
        \end{align*}
        
    \end{enumerate}
    
    \item
    
    Kepler's Third Law: $(M_1+M_2)P^2 = a^3$
    \begin{align*}
        a &= ((M_1+M_2)P^2)^{1/3}\\
        &= ((8)(0.0018)^2)^{1/3}\\
        &= 0.03 \text{AU}
    \end{align*}
    
    \item
    
    \begin{enumerate}
        \item 
        
        Mass-Luminosity Relationship: $L\propto M^4$
        \begin{align*}
            L &= (0.085)^4\\
            L &= 5.22*10^{-5}\\
            L &= (100)^4\\
            L &= 10^8.
        \end{align*}
        This range is between $5.22*10^{-5}$ to $1*10^8$ times as luminous as the sun.
        
        \item
        
        Mass-Luminosity Relationship: $L\propto M^4$
        \begin{align*}
            0.1 &= (M)^4\\
            M &= 0.56 \text{ times the mass of the sun}\\
            1000 &= (M)^4\\
            M &= 10^{3/4}\text{ times the mass of the sun}.
        \end{align*}
        
    \end{enumerate}
    
    \item
    
    Kepler's Third Law: $(M_1+M_2)P^2 = a^3$
    \begin{enumerate}
        \item 
        
        \begin{align*}
            (M_1+M_2)P^2 &= a^3\\
            (3.0027*10^{-5})(0.575)^2 &= a^3\\
            a &= 0.021 \text{AU}
        \end{align*}
        
        \item
        
        \begin{align*}
            (M_1+M_2)P^2 &= a^3\\
            (5.03*10^5)(0.00011)^2 &= a^3\\
            a &= 0.183 \text{AU}
        \end{align*}
        
        \item
        
        \begin{align*}
            (M_1+M_2)P^2 &= a^3\\
            (M_1+M_2)(0.25)^2 &= (0.47)^3\\
            (M_1+M_2) &= 1.66 \text{ solar masses}
        \end{align*}
        
        \item
        
        \begin{align*}
            (M_1+M_2)P^2 &= a^3\\
            (M_1+M_2)(1.82*10^{-3})^2 &= (0.02)^3\\
            (M_1+M_2) &= 2.42 \text{ solar masses}
        \end{align*}
        
        \item
        
        \begin{align*}
            (M_1+M_2)P^2 &= a^3\\
            (13)P^2 &= (5.2)^3\\
            P &= 3.29 \text{ years}
        \end{align*}
        
        
    \end{enumerate}
    
\end{enumerate}

\end{document}
