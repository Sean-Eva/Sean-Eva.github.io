\documentclass{article}
\usepackage[utf8]{inputenc}
\usepackage[english]{babel}
\usepackage{amsthm}
\usepackage{amssymb}
\usepackage{mathcomp}
\usepackage{amsmath}
\usepackage{natbib}
\usepackage{array}
\usepackage{wrapfig}
\usepackage{multirow}
\usepackage{tabularx}
\usepackage{multirow}
\usepackage{graphicx}

\newtheorem{ishaan}{Theorem}[section]
\newtheorem{lemma}{Lemma}[section]
\renewcommand\qedsymbol{$\blacksquare$}

\title{Homework 11}
\author{Sean Eva}
\date{October 2021}

\begin{document}

\maketitle

\begin{enumerate}
    \item 
    
    Volume of a sphere: $V= \frac{4}{3}\pi r^3$. Then,
    \begin{align*}
        V &= \frac{4}{3}\pi (\frac{100000}{2}*\frac{9.46*10^{17}}{1})^3\\
        V &= 4.434*10^{68} \text{cm}^3.
    \end{align*} Therefore, there are $4.434*10^{68}$ hydrogen atoms in the interstellar gas of our galaxy. Then the mass, $4.434*10^{68} * (1.67*10^{-27} \text{kg}) = 7.422*10^{41}\text{kg}$. This then means that there are $\frac{7.422*10^{41}}{1.989*10^{30}} = 3.732*10^{11}$ solar masses of hydrogen in our galaxy.
    
    \item
    
    Stefan-Boltzmann Equation: $\frac{L_2}{L_1} = (\frac{R_2}{R_1})^2(\frac{T_2}{T_1})^4$. Then,
    \begin{align*}
        1400 &= (R_2)^2 (\frac{480}{5800})^4\\
        R_2 &= \sqrt{\frac{1400}{(\frac{480}{5800})^4}}\\
        R_2 &= 5463 \text{ R}_\odot.
    \end{align*}This star has a radius of about $5463 \text{ R}_\odot$.
    
    \item
    
    Jean's Length: $R_J = \sqrt{\frac{15*k*T}{4*\pi*G*\mu*m_H*\rho_0}}$. Then,
    \begin{align*}
        R_j &= \sqrt{\frac{15(1.38*10^{-23})(110)}{4\pi (6.67*10^{-11})(1)(1.67*10^{-27})((1.67*10^{-27})(1.22*10^6))}}\\
        R_j &= \frac{2.83*10^{18}}{1.496*10^{14}}\\
        R_j &= 18917\text{ AU}.
    \end{align*}
    Freefall Time: $T_{ff} = \sqrt{\frac{3\pi}{32G\rho_0}}$. Then,
    \begin{align*}
        T_{ff} &= \sqrt{\frac{3\pi}{32(6.67*10^{-11})(1.67*10^{-27})(1.22*10^6)}}\\
        T_{ff} &= \frac{1.47*10^{15}}{31557600}\\
        T_{ff} &= 4.658*10^{7} \text{ years}
    \end{align*}
    
\end{enumerate}

\end{document}
